\section{Introduction}
\subsection{Background}
Medical professionals collect \gls{ohd} in \glspl{ehr} at various points of care in a patient’s trajectory, to support and enable their work \cite{Cowie_2016}. The patient profiles found in \glspl{ehr} are diverse and longitudinal, composed of demographic variables, recordings of diagnoses, conditions, procedures, prescriptions, measurements and lab test results, administrative information, and increasingly omics \cite{Ohdsi2020-vf}.\par
Having served its primary purpose, this wealth of detailed information can further benefit patient well-being by sustaining medical research and development. This could mean improving the development life-cycle of health informatics (HI), the predictive accuracy of machine learning (ML) algorithms, or enabling discoveries in research concerning clinical decisions, triage decisions, inter-institution collaboration and HI automation \cite{Rudin_2020}. Big health data is the underpinning of two prime objectives of precision medicine: individualization of patient interventions and inferring the workings of biological systems from high-level analysis \cite{Capobianco2020}. However, the private nature of patient-related data, and the growing widespread concern over its disclosure, hampers dramatically the potential for secondary usage of \gls{ohd}.\par

Anonymization techniques are used to hinder the misuse of sensitive data. This means a costly and data-specific cleansing process, enhancing privacy to the detriment of data utility. \todo{ref} These techniques are fallible and do not prevent reidentification. To address these drawbacks, alternative modes for sharing sensitive data is an active research area, including privacy-preserving analytics and distributed learning. Although promising, these approaches come with limitations and their feasibility has yet to be demonstrated.\par
These conditions restric access to \gls{ohd} to professionals with academic credentials and financial resources. The use of OHD by all other health data-related occupations is blocked, along with the downstream benefits. For example, software developers rarely have access to the data at the core of the health informatics solutions they are developing.
\subsection{Synthetic data}
An alternative to traditional privacy-preserving methods is to produce full \gls{sd} with methods to build these models including knowledge-driven and data-driven modelling \cite{Kim_2017}. Knowledge-driven modelling involves a complex theory-based process to define a simulation process representing the causal relationships of a system. The Synthea \cite{Walonoski_2017} synthetic patient generator is one such simulation model, in which predefined states, transitions, and conditional logic produce patient trajectories. The parameters of the Synthea model are taken from aggregate population-level statistics of disease progression and medical knowledge. A knowledge-based approach such as Synthea depends on prior knowledge of the system, and most importantly how much we can understand about it \cite{Kim_2017}. When modelling complex systems, simplifications and assumptions are inevitable, leading to inaccuracies. For example, relying on population-level statistics does not produce models capable of reproducing heterogeneous health outcomes \cite{Chen_2019}.\par
In data-driven modelling techniques, a representation of the data is inferred from a sample distribution. There exist numerous statistical modelling approaches to produce \gls{sd}, but the modelling processes are based on intrinsic assumptions about the data, the representational power is bound to correlations that are intelligible to the modeler or are prone to obscure inaccuracies. \gls{sd} generated by these models tends to possess low utility \cite{Rankin2020}. In the ML field, generative models learn to represent an estimate of the multimodal distribution, from which synthetic samples can be drawn \cite{goodfellow2016nips}. \Gls{gan} \cite{NIPS2014_5423} have recently emerged as a groundbreaking approach to efficiently learn generative models that produce realistic \gls{sd} using \gls{nn}. \gls{gan} algorithms have rapidly found a wide range of applications, such as data augmentation in medical imaging \cite{Kadurin_2017}.\par
The potential impacts of \gls{gan} to healthcare and science are considerable, some of which have been realized in fields such as medical imaging \cite{Yi_2019}. However, the application of \gls{gan} to \gls{ohd} seems to have been lagging \cite{Xiao_2018_chall}. Certain characteristics of \gls{ohd} could serve to explain the relatively slow progress. Primarily, algorithms developed for images and text in other fields were easily re-purposed for medical equivalents. However, \gls{ohd} presents a unique complexity in terms of multi-modality, heterogeneity, and fragmentation \cite{Xiao_2018_chall}. In addition to this, evaluating the realism of synthetic \gls{ohd} is intuitively complex, a problem that still burdens \gls{gan} in general. Nonetheless, interesting \gls{gan} solutions to the challenges posed by \gls{ohd} have been developed \cite{esteban2017real,Che_2017,Choi2017-nt,yahi2017generative}.