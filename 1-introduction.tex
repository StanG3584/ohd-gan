\glsresetall
\section{Introduction}
    \subsection{Background}
        Medical professionals collect \gls{ohd} in \glspl{ehr} at various points of care in a patient’s trajectory, to support and enable their work \cite{Cowie_2016}. The patient profiles found in \glspl{ehr} are diverse and longitudinal, composed of demographic variables, recordings of diagnoses, conditions, procedures, prescriptions, measurements and lab test results, administrative information, and increasingly omics \cite{Ohdsi2020-vf}.\par
        Having served its primary purpose, this wealth of detailed information can further benefit patient well-being by sustaining medical research and development. That is to say, improving the development life-cycle of \gls{hi}, the predictive accuracy of \gls{ml} algorithms, or enabling discoveries in research on clinical decisions, triage decisions, inter-institution collaboration and \gls{hi} automation \cite{Rudin_2020, Rankin2020}. Big health data is the underpinning of two prime objectives of precision medicine: individualization of patient interventions and inferring the workings of biological systems from high-level analysis \cite{Capobianco2020}. However, the private nature of patient-related data, and the growing widespread concern over its disclosure, hampers dramatically the potential for secondary usage of \gls{ohd} for legitimate purposes.\par
        
        Anonymization techniques are used to hinder the misuse of sensitive data. This implies a costly and data-specific cleansing process, and the unavoidable trade-off of enhancing privacy to the detriment of data utility. \todo{ref} These techniques are fallible and do not prevent reidentification. In fact, it has been demonstrated that no polynomial time \gls{dp} algorithms can produce \gls{sd} preserving all relations of the real data, even for simple relations such as 2-way marginals \cite{Ullman2011}. To address these drawbacks, alternative modes for sharing sensitive data is an active research area, including privacy-preserving analytic and distributed learning. Although promising, these approaches come with limitations and their feasibility has yet to be demonstrated. Regardless, distributed models are vulnerable to a variety of attacks, for which no single protection measure is sufficient as research on defense is far behind attack \cite{enthoven2020overview, Gao2020}.The process of \gls{dp} may also \par
        These conditions restrict access to \gls{ohd} to professionals with academic credentials and financial resources. The use of OHD by all other health data-related occupations is blocked, along with the downstream benefits. For example, software developers rarely have access to the data at the core of the \gls{hi} solutions they are developing, or educators lack appropriate examples \cite{laderas_teaching_2018}.
        
    \subsection{Synthetic data}
        An alternative to traditional privacy-preserving methods is to produce full \gls{sd}. Methods to produce \gls{sd} can either be categorized as either theory-driven (theoretical, mechanistic or iconic) and data-driven (empirical or interpolatory) modelling \cite{Kim_2017, Hand2019}. Theory-driven modelling involves a complex knowledge-based attempt to define a simulation process or a statistical model representing the causal relationships of a system \cite{Yousefi2018-dy, Kansal2018-dx}. The Synthea \cite{Walonoski_2017} synthetic patient generator is one such model, in which state transition models\footnote{Probabilistic model composed of predefined states, transitions, and conditional logic.} produce patient trajectories. The model parameters are taken from aggregate population-level statistics of disease progression and medical knowledge. Such a knowledge-based model depends on prior knowledge of the system, and most importantly how much we can intellect about it \cite{Kim_2017, Bonnery2019-ug}. On one hand theory-based modelling aims at understanding and offers interpretability, on the other when modelling complex systems, simplifications and assumptions are inevitable, leading to inaccuracies or reduced utility \cite{ranHand2019, Rankin2020}. In fact, relying on population-level statistics does not produce models capable of reproducing heterogeneous health outcomes \cite{Chen_2019}.\par
        
        Data-driven modelling techniques infer a representation of the data from a sample distribution, in an attempt to summarize or describe it \cite{Hand2019}. There exist numerous statistical modelling approaches to produce \gls{sd}, but the techniques are based on intrinsic assumptions about the data. The representational power is bound to correlations that are intelligible to the modeler, being prone to obscure inaccuracies. \gls{sd} generated by these models tends to hit a ceiling of utility \cite{Rankin2020}. In the ML field, generative models learn an approximation of the multi-modal distribution, from which synthetic samples can be drawn \cite{goodfellow2016nips}. \Gls{gan} \cite{NIPS2014_5423} have recently emerged as a groundbreaking approach to efficiently learn generative models that produce realistic \gls{sd} using \gls{nn}. \gls{gan} algorithms have rapidly found a wide range of applications, such as data augmentation in medical imaging \cite{Yi2019, Wang2020, Zhou2020}.\par
        
        The potential impacts of \gls{gan} to healthcare and science are considerable \cite{Rankin2020}, some of which have been realized in fields such as medical imaging. However, the application of \gls{gan} to \gls{ohd} seems to have been lagging \cite{Xiao_2018_chall}. Certain characteristics of \gls{ohd} could serve to explain the relatively slow progress. Primarily, algorithms developed for images and text in other fields were easily re-purposed for medical equivalents of the data types. However, \gls{ohd} presents a unique complexity in terms of multi-modality, heterogeneity, and fragmentation \cite{Xiao_2018_chall}. In addition to this, evaluating the realism of synthetic \gls{ohd} is intuitively complex, a problem that still burdens \gls{gan} in general. Nonetheless, in 2017 the first few attempts at \glspl{gan} for \gls{ohd} were published \cite{esteban2017real,Che_2017,Choi2017-nt,yahi2017generative}. We aimed to investigate if the field continued to expand following these first few examples, and if so to gain an comprehensive understanding of methods and approaches to the problem.