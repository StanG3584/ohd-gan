\begin{table}
    \begin{tabular}{@{}p{0.1\textwidth} p{0.7\textwidth}p{0.2\textwidth}@{}}\toprule
    Dataset & Summary & Link\\\midrule
    
    SPRINT Clinical Trial Data \cite{wright2016randomized} 
    & \begin{description} \item[Patients] Single-blind treatment trial of hypertensive patients.\item[Size] 6502 participants \item[Observations] 12 measurements (RZ, 1M, 2M, 3M, 6M, 9M, 12M, 15M, 18M, 21M, 24M, and 27M)\item[Features]Systolic blood pressure, diastolic blood pressure, and the count of medications prescribed\end{description} 
    & \href{https://challenge.nejm.org/pages/home}{SPRINT Data Analysis Challenge}\\
    
    Coalition Against Major diseases Online data Repository for AD \cite{Neville_2015} 
    & \begin{description} \item[Patients] Unified clinical trial database for Alzheimer’s disease \item[Size] 1909 patients \item[Observations] 18 months, at 3 month intervals. \item[Features] Longitudinal trajectories of 44 categorical, ordinal and continuous features \end{description} 
    & \href{https://c-path.org/programs/dcc/projects/alzheimers-disease/coalition-against-major-diseases-consortium-database-camd-admci/}{Critical Path Institute (C-Path)}\\
    
    Columbia Open Health Data (COHD) \cite{Ta_2018} & \begin{description} \item[General] EHR derived from Columbia University Irving Medical Center’s Observational Health Data Sciences and Informatics (OHDSI) database \item[Cohort] Lifetime dataset: 5,364,781 patients, 5-year dataset: 1,790,431 patients from 2013 to 2017 \item[Observations] Lifetime: 36,578 concepts 5-year: 29,964 concepts \item[Features] Lifetime: 11,952 conditions, 12,334 drugs, and 10,816 procedures, 5-year: 10,159 conditions, 10,264 drugs, and 8,270 procedures \item[privacy] Exclusion of rare concepts (count < 10) and Poisson randomization \end{description} & \href{http://cohd.io/}{COHD, a RESTful web API}\\

    Philips eICU \cite{pollard2018eicu} & \begin{description} \item[General]  Compiled from records shared by hundreds of ICUs across the United States, and is managed by the Philips eICU Research Institute \item[Cohort] 200,859 patient unit encounters for 139,367 unique patients admitted between 2014 and 2015  \item[Observations] Stratified random sample of patient index stays based upon hospital\item[Features] Vital signs, laboratory measurements, medications, APACHE components, care plan information, admission diagnosis, patient history, time-stamped diagnoses \item[privacy] HIPAA provisions, removal of all protected health information (PHI), such as personal numbers (e.g. phone, social security), addresses, dates, and ages over 89 \end{description} & \href{https://physionet.org}{Physionet \cite{Goldberger_2000}}\\
    
    Multiparameter Intelligent Monitoring in Intensive Care (MIMIC-III v1.4) \cite{Johnson_2016} & \begin{description} \item[General] Health-related data associated with patients who stayed in critical care units of the Beth Israel Deaconess Medical Center \item[Cohort] Forty-thousand patients \item[Features]demographics, vital sign measurements made at the bedside (~1 data point per hour), laboratory test results, procedures, medications, caregiver notes, imaging reports, and mortality (including post-hospital discharge) \item[privacy] Deidentified in accordance with Health Insurance Portability and Accountability Act (HIPAA) standards using structured data cleansing and date shifting \end{description} & \href{https://mimic.physionet.org}{MIMIC Physionet} \cite{Goldberger_2000}\\
    
    Vanderbilt University Medical Center Synthetic Derivative \cite{Roden_2008} & \begin{description} \item[General] Vanderbilt's collection of DNA samples coupled to a deidentified image of the Vanderbilt electronic medical record system, termed the "Synthetic Derivative." \item[Cohort]2.2 million patients \item[Features] clinical and demographic information, such as ICD 9 codes, CPT procedure codes, medications, lab values, age and gender \item[privacy]Cleared of HIPAA identifiers with an error rate of ~0.01\% \end{description} & \href{https://victr.vumc.org/biovu-description/}{BioVU}\\
    
    UC Irvine Machine Learning Repository \cite{Dua:2019} & \begin{description} \item[General] About 13 health related datasets from 1987 to 2020 \item[Cohort] Ranges from about 30 to 100K instances \item[Features] Datasets ranging from 10 to 200 categorical and/or numerical dimensions and time-series. \item[privacy] See relevant dataset. \end{description} & \href{http://archive.ics.uci.edu/ml/index.php }{UCI ML repository}\\
    
    Ward2ICU \cite{severo2019ward2icu} & \begin{description} \item[General] Electronic Health Records of patients from Hospital Mater Dei, a tertiary hospital, located in Belo Horizonte, Brazil \item[Cohort] Adult patients with an average age of 40, between the years of 2014 and 20 \item[Features] 25 vitals, of which 5 are currently available, 20 samples per patient \end{description} & \href{https://arxiv.org/abs/1910.00752}{ArXiv}\\
    
    SEER Cancer Statistics Review (CSR) \cite{noone2018cronin}  & \begin{description} \item[General] Annual NCI report of the most recent cancer incidence, mortality, survival, prevalence, and lifetime risk statistics. \item[Cohort] 11,135,91 patient cases, covering approximately 34\% of the US population \item[Features] Demographics, primary tumor site, tumor morphology, stage at diagnosis, and first course of treatment. \item[privacy] \end{description} & \href{https://seer.cancer.gov/data/access.html}{SEEr Incidence database}\\
    
    PREAGRANT \cite{Fasching_2015} & \begin{description}  \item[General] German-wide clinical study for breast cancer research \item[Cohort] \emph{Ongoing study, estimated:} 3,500 patients with locally advanced, inoperable/metastatic breast cancer. 10,000 patients with breast cancer in the neoadjuvant and adjuvant (early breast cancer) setting. \item[Features] Include age, tumor mass, grading, time from primary diagnosis to metastasis, and the site of metastasis. Multiple blod biomarkers which were the focus of the study. \item[privacy] Monitoring of all data is done using a professional query verification and source data verification system. \end{description} & Seemingly not plublicly available. Correspondance address: \href{mailto:peter.fasching@uk-erlangen.de}{peter.fasching@uk-erlangen.de} \\
    
    New Zealand National Minimum Dataset (hospital events) \cite{events} & \begin{description} \item[General] National collection of public and private hospital discharge information, including clinical information, for inpatients and day patients.    Submitted electronically in an agreed format by public hospitals since 1993. \item[Cohort] Approximatly 3M discharges for the the years 2016/17 alone \cite{2017} \item[Features] Dense administrative features such as patient age, gender, and length of stay, as well as sparse features such as diagnosis codes. \item[privacy] Information available to the general public is of a statistical and non-identifiable nature. \end{description} & \href{https://www.health.govt.nz/nz-health-statistics/access-and-use/data-request-form}{Data request form}\\
    
    Sutter Palo Alto Medical Foundation (PAMF) \todo{find more info about this data} \cite{Choi2017-nt} & consists of 10-years of longitudinal medical records of 258K patients  &\\
    
    heart failure study dataset from Sutter \cite{Choi2017-nt} & which consists of 18-months observation period of 30K patients. & \\
    
    
    \bottomrule
    \end{tabular}
\end{table}