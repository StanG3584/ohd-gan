\section{Discussion}


\subsection{Applications of GANs for health data and innovation}

Overall, the published \gls{gan} algorithms for \gls{ohd} provided equivalent or superior performance versus the statistical modeling-based methods against which they were benchmarked. Importantly, their capabilities are highly relevant to the medical field: domain translation for unlabeled data, conditional sampling of minority classes, data augmentation, learning from partially labeled or unlabeled data, data imputation, and forward simulation of patient profiles. While some of these claims are overoptimistic or lack convincing evidence, they paint an encouraging picture for the value of synthetic \gls{ohd} and the transformative effect it could have on healthcare initiatives and scientific progress.\par

The ongoing Covid-19 pandemic has brought unprecedented levels of cooperation between scientists from around the world. The urgency of obtaining data has highlighted in on difficult terms the need for novel ways of sharing and generating data \cite{bandara_improving_2020, Cosgriff_2020}. Global concerted efforts were highly successful, but also required adaptation, with some proposing exemptions from the GDPR \cite{mclennan_covid-19_2020}. Data sharing was limited to aggregate counts, rather than at the patient level, limiting the depth of analyses. \par

In the beginning of an epidemic, the scarcity of data can be compensated with synthetic data. This was attempted by Synthea \cite{Walonoski_2017} in the early months of the pandemic, with humble results, nonetheless they were used in many online challenges, hackathons, and conferences. 
\begin{quote}
    The authors state that if one takes "[...] Field Marshall Moltke’s notion of “no plan survives contact with the enemy” as true and expands the scope to modeling and simulation, then we might say that “no model survives contact with reality.” \cite{walonoski_synthea_2020}. We would argue that \glspl{gan} grow stronger in contact with reality. 
\end{quote}
Generative models refine their generally improve their representation as more data is provided and could be combined with current methods of forecasting. When the amount of ground truth data is small, semi-supervised learning simulations can improve the performance of predictors \cite{dahmen_synsys_2019}. Domain translation, as demonstrated in \gls{radialgan}, would be exceptionally useful to combine datasets from such disparate localities. In a recent publication, two different data augmentation techniques provided a significant increase in sensitivity and specificity for the detection of COVID-19 infections, one of which producing \gls{sd} with a \gls{gan} \cite{Sedik2020-tx}.

\subsection{Challenges posed by OHD}

The challenges posed by health data for \glspl{gan} are obvious,a number of recurrent factors influence the outcome of efforts to develop them. These problems are not limited to generative algorithms, but also \gls{ml} in general. Uniquely for generative models, multi-modality is one aspect that caused the most trouble in achieving a stable training procedure. At the outset, preventing \gls{mode-collapse} attracted the most research efforts, in addition to data combinations of categorical and real-valued features. A rapid succession of efforts aimed at improving \gls{medgan} by incorporating the latest machine learning techniques showed continued improvements. However, taken as a whole the efforts were haphazard in their methods and metrics. Often yielding unsurprising results, considering the techniques were known to improve performance across a broad range of applications. This is expected in a new field of application, and more concerted efforts to systematically approach the problems should progressively consistency.\par

\subsubsection{Feature engineering}
We observed that majority of methods included in the review made use of heavily transformed representations of patient records. This is in part due to the inconvenient properties of health data, such as missingness. However, it is somewhat apparent that the main motive is to accommodate existing algorithms.\par
Along with demographic variables, \gls{ohd} data mostly takes the form of triples composed by (1) a timestamp, (2) a medical concept and (3) the recorded value. Their count is different for each patient, irregular intervals between each triple and the number of possible values in a dimensions can be huge. Moreover, there are generally multiple episodes of care, each with a different cause. These properties are not typically considered practical for machine learning. \par
At varying degrees, depending on the transformations, information is being lost or bias is introduced. For example, when data are reduced by aggregation to one-hot encoding, the complex relationships found in medical data are, for the most part eliminated. Similarly, information is lost when forcing real-valued time-series into a regular representation, by truncating, padding, binning or imputation. Moreover, it is highly unlikely that the data is missing at random, introducing the potential for bias when a large part of the real data is rejected on this basis, or the medical codes are truncated to their parent generalizations \cite{Zhang2020, Choi2017-nt}. In brief, loss of information content is being dismissed, as opposed to the more tricky alternative of developing algorithms to accommodate the data.\par

\subsubsection{Innovation to adoption}
A number of interesting innovations were nonetheless demonstrated, and progress has good momentum. Their application and adoption will undoubtedly be more sluggish, as has been the case with predictive \gls{ml}. For good reason, the bar is set high in terms in demonstrating consistent outcomes and ensuring patient safety. While the problem of \gls{mode-collapse} has been alleviated, evidence has yet to be provided with regards to ensuring that the finer details of the distribution are estimated with sufficient granularity to produce realistic patient profiles. In this regard, conditional training methods have led to improvements. For example, when labels corresponding to sub-populations or classes are used to condition the generative process. \citeauthor{Zhang2020} showed that conditioned training with categorical labels, in this case age ranges, improves utility for small datasets \cite{Zhang2020}. As described in Section \ref{noauto}, \gls{heterogan} further introduces constraint-based loss. Based on the distribution of individual features and utility-based metrics, the authors argue that the bias intrinsic to their methods has not led to undesirable bias or side-effects in other aspects of the learned distribution. However, the constraints were strict and would be hard to scale (see Section \ref{sec:knowledge} for an alternative approach). 

\subsection{Evaluation metrics and benchmarking}
Consistent behavior and reproducible results will be required to expect any significant adoption. In regards to evaluation, it is manifest that the choice of optimal metrics and indicators is still being explored. The blatant observation is that the efforts are far from consistent or systematic. As an example, competing methods are often compared with different metrics or with contradictory results in different datasets \cite{baowaly_2019_IEEE,baowaly_2019_jamia,Camino2018-re,Choi2017-nt,Zhang2020}. Overall, none of the evaluation metrics addressed the concept of realism in synthetic data\par

\subsection{Qualitative evaluation}
Qualitative evaluation, in its current form, provides little evidence. For medical experts, these representations are meaningless. As such, the results of qualitative evaluation often state that synthetic data is indistinguishable from the real data \cite{Choi2017-nt,Wang_2019}. It is doubtful that they could in fact be distinguished. \citeauthor{esteban2017real} found that participants avoided the median score and were not confident enough to choose either extreme \cite{esteban2017real}.\par

In their evaluation of \gls{medgan}, \cite{yale:hal-02160496} argue that the positive resemblance of plotted feature distributions is due to the fact that the model's architecture tends to favor reproducing the means and probabilities of each diagnosis column. They note that synthetic data contains samples with an unusually high number of codes, which is not apparent in the plots. Their hypothesis is that these samples are used by the algorithm to discharge the rare medical codes with weak correlation, in an effort to balance the distributions. However, they stated in their experiments that comparing \gls{pca} plots of real and synthetic data for was nonetheless insightful to get an impression of their behavior \cite{Yale_2020}.\par

\citeauthor{Ledesma2016-hn} describe the problem of medical data representation and visualization learnedly, from information quality and usefulness, timescales and perception, to user satisfaction and aesthetics. The evaluation of their solution is extensive, detailed and rigorous, done according to the well known Nielsen's heuristics for Human-Computer Interaction \cite{nielsend}. Of which Principles \#2, for example: "\textbf{Match between system and the real world}: The system should speak the users' language, with words, phrases and concepts familiar to the user, rather than system-oriented terms. Follow real-world conventions, making information appear in a natural and logical order.". \footnote{Interested readers can find the remainder here \href{https://www.nngroup.com/articles/ten-usability-heuristics/}{Ten Usability Heuristics}.} While this may seem like total digression towards graphic design, it is rather to illustrate the complexity of aspects to be considered before representing data in a evaluation task.

\subsubsection{Quantitative evaluation}
Reproducing aggregate statistical properties is rather unconvincing evidence that a model has learned to reproduce the complexity of patient health trajectories. In some cases the statistical metrics may be contradictory, such as when the ranking of medical frequencies in the data are wrong, but augmentation leads to improved performance \cite{Che_2017}. \citeauthor{Choi2017-nt} found that although the synthetic sample seemed statistically sound, it contained gross errors such as gender code mismatches and suggested the use of domain-specific heuristics \cite{Choi2017-nt}. \gls{heterogan} was an encouraging step in this direction, but it may be difficult to scale. The idea of incorporating knowledge-based constraints in the otherwise naive \gls{gan} is in fact gaining attention (See Section \ref{sec:knowledge} \par

Utility-based metrics do overall provide a more solid evaluation of data quality. However, they only confirm the value of the data according to a narrow context. They are indicative of realism so far as a patient's state is indicative of a medical outcome. Moreover, they do not provide any insight about the validity of the relations found in a patient record and its overall consistency. While such consideration was found in sparingly in the publications, extensive research available on the subject of medical information representation. The complexity of health data and its variety make it a considerable, but captivating challenge.\par

\section{Suggestions of requirements for OHD-GAN development}

\subsection{Models of appropriate scope for one's claims}\label{sec:basic}
Overall, evaluation methods were superficial or uni-dimensional. Finding convincing and robust evaluation metrics for synthetic health data is an open issue. Weak metrics become a prominent issue when the learning task is broad, loosely defined, constructed for the sole purpose of evaluation, or the scope of application is too large. The difficulty of explaining or validating the realism of data representing a patient, often longitudinal and which factors deferentially contribute to disease characterization makes the assessment of synthetic data ambiguous, thus demanding stronger evidence to claims.\par

\footnotesize
\tcbset{enhanced,before skip=1cm, nobeforeafter, width=0.5\linewidth}
\begin{tcolorbox}[arc=0mm, 
    colback=cadmiumgreen!10!white, 
    coltext=cadmiumgreen!90!black,  
    colframe=cadmiumgreen!90!black,
    colbacktitle=cadmiumgreen!80,
    leftrule=3mm,
    rightrule=0mm, 
    toprule=0mm, 
    bottomrule=0mm, 
    box align=top]

Modelling efforts for OHD-GAN should be limited in scope to develop robust algorithms for a single data type or modality. 
- This makes qualitative evaluation by visual inspection from experts possible and meaningful.
- The behaviour of the model can be assessed straightforwardly. 
- Conditional models are easier to develop. 
The evaluation metrics should not be defined solely for the purpose but from a peer-reviewed healthcare publication.

\end{tcolorbox}
\hfill
\begin{tcolorbox}[tcbox width=auto, 
    arc=0mm, 
    colback=white, 
    coltext=cadmiumgreen, 
    boxrule=0pt, 
    colframe=white,
    box align=top]
    
\epigraph{\textit{A baby learns to crawl, walk and then run.  We are in the crawling stage when it comes to applying machine learning.}}{\textit{Dave Waters}}

\end{tcolorbox}
\normalsize%
%
\subsection{Data-driven architecture}\label{sec:archi}
The architecture of \gls{ohd}-GAN should be engineered to match the data, not the other way around. Data with minimal transformations, to the extent possible. In addition to preventing information loss, this ensures models will reflect the real generative process. Such models are more likely to further our understanding about them and the biological drivers. Furthermore, the learned statistical distribution is inevitably more meaningful and interpretable, facilitating applications in the healthcare domain and supporting the inference of insights.\par

Deep architectures are based on the intuition that multiple layers of nonlinear functions are needed to learn complicated high-level abstractions \cite{Bengio_2009}. \gls{cnn} capture patterns of an image in a hierarchical fashion, such that in sequence, each layer forms a representation the data at a higher level of abstraction. This type of data-oriented architecture has led to impressive performance for \gls{cnn} and image data. Health data presents a different, analogous multi-level structure. As an illustration, a predictive algorithm developed in a hierarchical structure was shown to form representations of \gls{ehr} that capture the sequential order of visits and co-occurrence of codes within a visit. It led to improved predictor performance, and also allowed for meaningful interpretation of the model \cite{choi2016multi}. Similarly, models of time-series based on a continuous time representation, such as \glspl{eeg} and \glspl{ecg} found in \gls{ehr} data, have shown improved accuracy over discrete time-representations \cite{rubanova2019latent,de2019gru}. Those interested in \gls{gan} for wavelike data will find many examples \cite{Delaney2019,Golany2019,Ye2019,Wang2019d,Singh2020,Aznan2019,Hartmann2018}.  Creative adaptations of the data for existing architectures have provided surprising results. For example, \gls{ohd} input into a CNN were transformed to image(bitmaps) in which the pixels encoded the information \cite{Fukae2020}.
As we have seen, \gls{ohd-gan} are not exclusively used to produce "fake" patients, but also in cases where the synthetic data is intended to be representative of the patient. Common examples are translating patient between states, or producing counterfactuals. It would be interesting to if see if combining \gls{gan} with what is know as Evolutionary computing could produced valuable results. We can think of a \gls{gan} transforming the patient data to an alternative state, after which the evolutionary algorithms would optimize this new state in a continuous fashion, as new data about the patient becomes available.

\footnotesize
\tcbset{enhanced, before skip=1cm, nobeforeafter, width=0.5\linewidth}
\begin{tcolorbox}[
    arc=0mm, 
    colback=cadmiumgreen!10!white, 
    coltext=cadmiumgreen!90!black,  
    colframe=cadmiumgreen!90!black,
    colbacktitle=cadmiumgreen!80,
    leftrule=3mm,
    rightrule=0mm, 
    toprule=0mm, 
    bottomrule=0mm, 
    box align=top]
    
The architecture of \gls{ohd}-GAN should be engineered to match the data, not the other way around. Data with minimal transformations, to the extent possible. In addition to preventing information loss, this ensures models will reflect the real generative process. Such models are more likely to further our understanding about them and the biological drivers. With deeper understanding, novel architecture of higher complexity will be engineered. Furthermore, the learned statistical distribution is inevitably more meaningful and interpretable, facilitating applications in the healthcare domain and supporting the inference of insights. 

\end{tcolorbox}
\hfill
\begin{tcolorbox}[tcbox width=auto, 
    arc=0mm, 
    colback=white, 
    coltext=cadmiumgreen, 
    boxrule=0pt, 
    colframe=white,
    box align=top]

\epigraph{Torture the data, and it will confess to anything.}{\textit{Ronald Coase}}

\end{tcolorbox}
\normalsize%
%
\subsection{Forcing, disciplining or guiding \label{sec:knowledge}}
To build statistical models we define rules that they are forced to optimize when learning. On the other hand, \glspl{gan} are given free range in a space of possibilities and are disciplined for exploring certain areas, but are provided no explanation. We build enormous models and let them fight back and forth in a mim-max battle that goes on forever, denying them our valuable knowledge. The idea of introducing human knowledge in the otherwise naive training process has gained some attention. It has recently been demonstrated that it can be introduced into the training procedure of GANs, blurring the lines between knowledge-driven and data-driven algorithms.\par

Posterior regularization us usually used to impose constraints on probabilistic models, but \glspl{gan} lack the necessary Bayesian component. In the field of \gls{rl}, a mathematical correspondence between \gls{ps} and \gls{rl} led to the probabilistic \gls{pr} framework \gls{irl} that seeks to learn a reward function from expert demonstrations. This was followed by approaches capable of learning both the reward function and the policy \cite{finn2016guided,fu2018learning}. \citeauthor{Hu2018} then demonstrated a correspondence between \glspl{rl} and \glspl{gan}. This allowed them to develop a \gls{gan} with a constraint-based learning objective \cite{Hu2018}. The constraints, seen as a reward function, can be learned by the model through an algorithm involving Maximum entropy. This means the known constraints can be input directly or partially and left to be learned automatically. The algorithm consistently improved the speed and quality of training, and accuracy on a few tasks. The approach is exemplified on an image translation task where images of people are transformed from one pose (ex. looking forward) to another (ex. head turned left). The constraint is provided by a pre-trained auxiliary classifier that assigns each pixel to a body part, and is adapted jointly with the \gls{gan}. The \gls{gan} is rewarded for preserving the mapping in the output image. A performance comparison against unconstrained and fixed-constraint models results in similar training loss and evaluation metric. However when evaluated by humans, the novel approach surpasses the other models on 77\% of test cases. \par

\subsection{Interpretability\label{sec:latent-space}}
Even though a few authors attempted to understand the behavior of their models, overall the subject was left largely unmentioned. It is imperative that future experimentation and publication give equal importance to the interpretation of their models and establishing means to do so. In the healthcare domain, black box machine learning models find little adoption, and synthetic data is most often met with dismissal to its validity. The task is not impossible, as for any other opaque system, and in fact experimental sciences in general. The simplest approach is to  provide input, observe the output, reformulate our hypotheses, and modify the input accordingly. Repeatedly, to convergence. Fortunately, in this case the internals workings are entirely available, tipping the balance between brute-force, and knowledgeable-driven exploration of the system. In addition, we believe "qualitative" evaluation by visual inspection has much greater potential, still to be defined. What better to define interpretation than a medical professional decoding the hidden relations in data visually. \par

In theory, the latent space is a lower-dimensional representation of basic concepts that should be directly interpretable. However, in practice these concepts are entangled over multiple nodes. In what is a preliminary, but encouraging proof-of-concept, \cite{lui2019-latent} explore how they can use perturbations to reveal patterns in a \gls{beta-vae} trained to capture brain structure in mice. By generating a collection of images from a dense interpolation of the latent space, they were able to examine the projective field of latent variables onto the pixels. They found zones of high variance that corresponded to biologically relevant areas. Reversing the experiment, they masked areas of the images and found that many latent factors were not activated by all regions of interest and had localized receptive fields. Whereas complex highly connected regions such as the hippocampus activated almost all latent factors. Curiously, the projective and receptive fields may not be aligned. Numerous other publications have shown that they capture meaningful properties and structure of the data, reducing complexity to a level that lends itself to interpretation \cite{Way2020, Koumakis2020}. In one instance involving transcription factor micro-array data, a close one-to-one mapping could be obtained from the last hidden layer, in addition to the higher level layers that related to biological processes in a hierarchical fashion \cite{chen2016-latentyeast}. Pushing the boundaries further, by correlating the output features of a GAN with the latent space dimensions allowed controllable semantic manipulation of the generated data \cite{Wang2020latent,Ding2020latent,Li2020latent}. However, a recent information-theoretic \gls{gan} simplified interpretation greatly by forcing the latent nodes to learn disentangled representations. In addition to adversarial loss, \gls{info-gan} also maximizes the mutual information between small numbers of latent nodes. The result is highly interpretable nodes that represent distinct concepts that can be easily influenced, or in some cases interpolate smoothly between features \cite{Chen2016c}.\par

\subsection{Benchmarking, a priority}
It became slowly obvious through the and secession of experiments, that there is a glaring problem of standardization of evaluation. New algorithms and application are being demonstrated at an increasing rate. On the contrary, standardized benchmarks, procedures to transform the data, and source has remained scarce, one can hardly compare the models objectively or nominate the best performances. Commendably, \citeauthor{Camino2020bench} are the first bring attention to this issue in a position paper that provides quantitative arguments. Notably the myriad of ways commonly used datasets are reprocessed, metrics that are not comparable, and hyperparater sweep results, for which no transformation code and optimal values are released and the lack of effort towards reproducibility will only reduce credibility of the field. On a positive note, we've compiled a list of the repositories which were made open-source in Table \ref{tab:5:sourcecode} and a list of the common dataset links can be found in Table \ref{tab:5:sourcecode}.\par
In this regard the replication of medical studies with synthetic data by \citeauthor{Yale_2020} substantiate the value of \gls{sd} for exploratory data analysis, reproducibility on restricted data and more generally education in scientific training \cite{Reiner_Benaim2020-lx}. Reproducing medical or clinical studies will be necessary to gain mainstream adoption of \gls{gan} produced \gls{sd} and dispel the scepticism it is generally met with. The medical domain is known for its slow pace in adopting new technologies and predictive \gls{ml} is still far from meeting its full implementation potential. Medical professionals care fore and foremost about the well-being of their patients and will they will only consider results obtained from synthetic data if they have the assurance that they are valid \cite{Rankin2020}.  A remarkable resource for the purpose of benchmarking is the clinical prediction benchmarks defined on the \gls{mimic} data by \citeauthor{harutyunyan_multitask_2019}. The tasks are clearly defined and the source code to process the data and the algorithms is available \cite{harutyunyan_multitask_2019}. We suggest comparing the accuracy of the predictive algorithms applied to the original data versus the synthetic data to be evaluated. However, concerted efforts and informal guidelines can be agreed upon should be on a regular schedule. We fully support the idea or organized challenges and hackahton proposed buy \cite{Camino2020bench} and suggest a progressive approach to realizing it.\par

\subsubsection{Ultra-open source, collaborative, publishing communities}
In a successful and educative experiment on collaborative writing, crowd-sourcing, a review was entirely writing in an open-source GitHub repository. Anyone willing to add their knowledge to the publication was welcome to do so, reaching 30+ authors in 20 countries. Every change proposal is requested for inclusion by a Pull Request, for which R2-3 approvals are necessary. Withing minutes, an automated deployment procedures (Github since then released Actions, requiring minimal coding), take care of verifying compliance to guidelines, citation management, DOI registration, and compilation of latex or Markdowdn. Withing minutes an up to date document is released, making the publication a contiguously up to date source of knowledge, than can be augmented in the web version with interactive code-books and figures.\par
Issues can be discussed in the appropriate channels, but most most importantly the nature of GitHub ensures attribution of work done, down to a single character. The authors also implemented immutable backup on the blockchain. Since then distributed storage and computation blockchains have have reached maturity and could store models, training artefacts, and data for competition at a trivial cost. As an alternative,  the \href{http://bit.ly/WandB-ML}{Weights and Biases (WandB)} platform is a fitting environment, worth a look even for individuals. The traditional publishers have long been touting a makeover of the publication system, changes are slow and trivial, whereas den centralized, person to person, systems have been transforming whole sectors faster than ever.\\

\section{Directions for future research}
\subsection{Building a patient model}
The ultimate goal for generative models of \gls{ohd} must be to develop an algorithm capable of learning an all encompassing patient model. It would then be possible to generate full \gls{ehr} records on demand, integrating genetic, lifestyle, environmental, biochemical, imaging, clinical information into high-resolution patient profiles \cite{Capobianco2020}. This is in fact the intention of the patient simulator Synthea. However, Synthea will eventually face a problem with scalability and the capacity of semi-independent state-transition models to coordinate in capturing long-range correlations.\par

Once basic models of health data, as described in Section \ref{sec:basic}, have been developed and validated, these can be progressively combined in a modular fashion to obtain increasingly complex patient simulators. Furthermore, having designed the architecture of these basic models on the underlying data in a way that is comprehensible, as described in \ref{sec:archi}, will facilitate the composition of more complex models. Inputs, outputs and parts of these models can be conditionally attached to others such that the generative process occurs in a way that reflects the real generative process.

\subsection{Evaluating complex patient models \label{sec:evaluation-cqm}}
Once more complex models are developed, the problem is again finding meaningful evaluation metrics of data realism. Capobiano et al. insist on the necessity for data performance metrics encompassing diagnostic accuracy, early intervention, targeted treatment and drug efficacy \cite{Capobianco2020}. In their publication exploring the validation of the data produced by Synthea, Chen et al. provide an interesting idea to achieve this \cite{Chen_2019}. Noting that the quality of care is the prime objective of a functional healthcare system, they suggest using \glspl{cqm} to evaluate the synthetic data. These measures "are evidence-based metrics to quantify the processes and outcomes of healthcare", such as "the level of effectiveness, safety and timeliness of the services that a healthcare provider or organization offers."(Chen 2019). High-level indicators such as \glspl{cqm} domain specific measures of quality, specifically designed for higher level or multi-modal representations of healthcare data. The constraints introduced in \gls{heterogan} should be leverage to evaluate the realism of the synthetic data, rather than bias the generator training. Composing a comprehensive set of such constraints could possibly serve as a standardized benchmark.
At the individual level, Walsh et al. employ domain specific indicators of disease progression and worsening and compare agreement of the simulated patient trajectories with the factual timelines \cite{walsh2020generating}.\par
In addition to \gls{cqm}, we propose the use of the Care maps used by the Synthea model to simulate patient trajectories as evaluation metrics \cite{Walonoski_2017}. Care maps are transition graphs developed from clinician input and Clinical Practice Guidelines, of which the transition probabilities are gathered from health incidence statistics. While these allow the Synthea algorithm to simulate patient profile with realistic structure, they also prevent it from reproducing real-world variability. Conversely, while \glspl{gan} have the ability to reproduce the quirks of real data, they also lack the constraints preventing nonsensical outputs. As such, Care maps provide an ideal metric to check if the synthetic data conforms to medical processes.\par 
In fact, has been used before in a competition where participants were given synthetic data from finite state transition machines with know probabilities and tasked to build and learn models that would reproduce those of the original, unseen models. The participants according to the Perplexity metric. Commonly used in NLP, quantifies how well a probability distribution or probability model predicts a sample \cite{Verwer_2013}. We postulate that the Synthea models built with real-world probabilities would provide a unique and robust way to evaluate synthetic data according to the metric proposed above, among other means to utilize the state-transition in Synthea and their modularity.

\subsubsection{Opportunities and application to current events}
Synthetic and external controls in clinical trials are becoming increasingly popular \cite{Thorlund2020}. Synthetic controls refer to cohorts that have been composed from real observational cohorts or \gls{ehr} using statistical methodologies. While the individuals included in the cohorts are usually left unchanged, micro-simulations of disease progression at the patient level are used to explore long-term outcomes and help in the estimation of treatment effects \cite{Thorlund2020, Etzioni2002}. Synthetic data generated by \glspl{gan} could be transformative for the problem of finding control cohorts.\par
With the COVID-19 pandemic scientists have become increasingly aware of and vocal about the need for data sharing between political borders \cite{Cosgriff_2020,Becker_2020,McLennan_2020}. An obvious application is generating additional amounts of data in the early stages of the pandemic, potentially creating opportunities earlier. Synthetic is data not only an opportunity to facilitate the exchange of data, but also adjust the biases of samples obtained from different localities. Factors such as local hospital practices, different patient populations and equipment introduce feature and distribution mismatches \cite{Ghassemi2020}. These disparities can be mitigated by translation of \gls{gan} algorithms, such as \gls{cycle-gan} proposed by Yoon et al.