 \subsection{Data Types and Feature Engineering}
    No publications made use of \gls{ohd} in its initial form, patient records in \glspl{gan} composed of many related tables (Normal form). The complexiety of a model wouuld grow rapibldy when maintaining referential inegrty and statistics between multiple tables. The hierarchy by witch these would interact with each other conditionally is no less complicated (see discussion Section \todo{section reference with a mention of a few statistical solutions that faced a number of problems}. There are however published \gls{gan} algorithms made to consume normalized database in their original form. \todo In regards to \gls{ohd}, feature engineering was used to adapt the data to task requirements, or to a promising algorithms that fit the date characteristics. The data is transformed into one of four modalities: time series, point-processes, ordered sequences or aggregates described in Fig. \ref{tab:features}
    
    \begin{table}[htpb]
\footnotesize
\caption{Types of observational health data and features engineering}\label{tab:features}

\begin{tabularx}{\textwidth}{@{}p{0.15\textwidth}p{0.3\textwidth}p{0.3\textwidth}X@{}} \toprule
Type & Values and structure & Challenges & Features engineering\\ \midrule

\textbf{Time-series}\newline
\textit{Continuous}\newline 
\textit{Regular}\newline
\textit{Sporadic}
&\begin{minipage}[t]{0.3\textwidth}{
\begin{itemize}[leftmargin=*]  
    \item Timestamped observations 
    \item Continuous, ordinal, categorical and/or multi-categorical
    \item Recorded continuously by medical devices, following a schedule by medical professional, or when necessary
\end{itemize}}
\end{minipage}
&\begin{minipage}[t]{0.3\textwidth}{
\begin{itemize}[leftmargin=*]  
    \item Observations are often \gls{mar} across time end dimensions, erroneous, or completely absent for certain patients.
    \item Time-series of different concepts are often highly correlated and their influence on one another must be acccounted for.
 \end{itemize}}
 \end{minipage}
& Imputation coupled with training \newline Regular \newline Data imputation \newline Binning in into fixed-size intervals \newline Combination of binning and imputation \\

\textbf{Point-processes} 
&\begin{minipage}[t]{0.3\textwidth}{
\begin{itemize}[leftmargin=*]  
    \item Series of timestamped observations of one variable or medical concept per patient
\end{itemize}}
\end{minipage}
&\begin{minipage}[t]{0.3\textwidth}{
\begin{itemize}[leftmargin=*]  
    \item \todo{Intensity functions, paramtric models}
\end{itemize}}
\end{minipage}
& Series of events reduced to the time interval between each consecutive occurrence. \\ \todo 

\textbf{Ordered sequences} 
& \begin{minipage}[t]{0.3\textwidth}{
\begin{itemize}[leftmargin=*]  
    \item Ordered vectors representing one or more patients visits
    \item Medical codes associated with the diagnoses, procedures, measurements and interventions
\end{itemize}}
\end{minipage}
& Variable length\newline High-dimensional\todo\newline Long-tail distribution of codes 
& Sequences are projected into a trained embedding that preserves semantic meaning according to methods borrowed from NLP\\

\textbf{Tabular}\newline Denormalized\newline Relational
&\begin{minipage}[t]{0.3\textwidth}{
\begin{itemize}[leftmargin=*]  
    \item Medical and demographic variables aggregated in tabular format
    \item Continuous, ordinal, categorical and/or multi-categorical features
\end{itemize}}
\end{minipage}
& Medical history is aggregated into a fixed-size vector of binary or aggregated counts of occurrences and combined with demographic features.\\



\bottomrule
\end{tabularx}
\end{table}