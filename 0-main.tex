
\documentclass[10pt]{article}


\usepackage{fullpage, setspace, parskip, titlesec}
\usepackage[section]{placeins}
\usepackage[dvipsnames]{xcolor}
\usepackage{breakcites, lineno, hyphenat, rotating, etoolbox}
\usepackage[colorlinks]{hyperref}
\usepackage{authblk, graphicx}
\usepackage{latexsym, textcomp, longtable, tabulary, tabularx}
\usepackage{booktabs,array,multirow}
\usepackage{amsfonts,amsmath,amssymb}
\usepackage[utf8]{inputenc}
\usepackage[english]{babel}
\usepackage{xspace, caption}
\usepackage{glossaries}
\usepackage[automake]{glossaries-extra}
\usepackage[round]{natbib}
\usepackage[left=0.75in]{geometry}
\usepackage[ampersand]{easylist}
\usepackage{enumitem}
\usepackage{pifont, makecell}
\usepackage[disable]{todonotes}
\usepackage{float}
\usepackage{lscape}
\usepackage{tcolorbox}
\usepackage{titlesec}
\usepackage[default]{opensans}
\usepackage[T1]{fontenc}

\definecolor{darkblue}{RGB}{0,0,130}
\definecolor{darkgreen}{RGB}{0,80,20}


\hypersetup{
    colorlinks = true,
    linkcolor = darkblue,
    anchorcolor = blue,
    citecolor = darkgreen,
    filecolor = blue,
    urlcolor = blue
}

\bibliographystyle{unsrtnat}

\PassOptionsToPackage{hyphens}{url}

\renewenvironment{abstract}
 {{\bfseries\noindent{\abstractname}\par\nobreak}\footnotesize}
 {\bigskip}

\titlespacing{\section}{0pt}{*3}{*1}
\titlespacing{\subsection}{0pt}{*2}{*0.5}
\titlespacing{\subsubsection}{0pt}{*1.5}{0pt}

\providecommand\citet{\cite}
\providecommand\citep{\cite}
\providecommand\citealt{\cite}
\renewcommand\cite{\citep}

\AtBeginDocument{\DeclareGraphicsExtensions{.pdf,.PDF,.eps,.EPS,.png,.PNG,.tif,.TIF,.jpg,.JPG,.jpeg,.JPEG}}
\renewcommand\labelitemi{-}

\newglossary[odg]{oalgo}{old}{odn}{OHD-GAN Acronyms}
\setabbreviationstyle[acronym]{long-short}
\glsxtraddallcrossrefs
\makeglossaries


%%%%%%%%%%%%%%%%%%%%%%%%%%% Algorithms
\newglossaryentry{gumbel-gan}
{
        name=Gumbel-Softmax GAN,
        description={}
}

\newglossaryentry{arae}
{
        name=ARAE,
        description={Adversarially regularized autoencoders.}
}

\newglossaryentry{t-sne}{
    name=t-SNE,
    description={The t-Distributed Stochastic Neighbor Embedding clustering algorithm is a nonlinear dimensionality reduction technique commonly applied to high-dimensiona data. See \citet{maaten2008tsne}.}
}

\newglossaryentry{mode-collapse}
{
        name=mode collapse,
        description={The training procedure fails to converge, or converges to an undesirable local minima resulting in a lack of variety in the generated samples.}
}

\newglossaryentry{feed-forward}
{
        name=feed-forward network,
        description={Basic Neural Network in its simplest form.}
}

\newglossaryentry{mb-avg}
{
        name=Mini-batch averaging,
        description={Adaptation of mini-batch averaging to cope with mode collapse, see \cite{choi2017generating}}
}

\newglossaryentry{dom-tran}{
    name=domain translation,
    description={Transforming data points from one domain or category to another.}
}

\newglossaryentry{semi-sup}{name=semi-supervised, description={\todo{definition}
}}

\newglossaryentry{re-iden}{name=reidentification attack, description={\todo{definition}
}}

\newglossaryentry{dbio}{name=digital bio-markers, description={\todo{definition}
}}

\newglossaryentry{exploding}{name=exploding gradient,
    description={The gradients accumulate large amounts of error, destabilising or disabling the training procedure.}
}

\newglossaryentry{vanishing}{name=vanishing gradient, description={The gradients become null and the network can no longer be updated.
}}

\newglossaryentry{a-disclosure}{name=Attribute disclosure, description={\todo{definition}}}

\newglossaryentry{mem-inference}{name=Membership inference, description={\todo{definition}}}

\newglossaryentry{repro-rate}{name=Reproduction rate, description={\todo{definition}}}

\newglossaryentry{utility-metric}{name=utility-based metric, description={In a broad sense, any metric that measure the amount of work that can be done with the data}}

\newglossaryentry{iteff}{name=Individual Treatment Effects, description={\todo{definition}
}}
\newglossaryentry{pmed}{name=Personalized Medicine, description={\todo{definition}
}}

\newglossaryentry{}{name=, description={
}}

%% Missingness
\newglossaryentry{mar}{name=Missing at Random, description={Given a dataset with missing entries , the missingness depends only on the observed variables \cite{yoon2018imputation}.
}}

\newglossaryentry{mcar}{name=Missing Completly at Random, description={Given a dataset with missing entries, the missingness is not depedant on any of the variables, thus occurs completly at random \cite{yoon2018imputation}.
}}

%% Self and co-training semi-supervised training
\newglossaryentry{co-training}{name=co-training, description={The self-training and co-training methods use classifiers first trained on the portion of labelled data to predict the labels of unlabelled instances. The newly labelled samples with the highest confidence are added to the labelled set to retrain the classifiers. The process is repeated iteratively.}}

\newglossaryentry{self-training}{name=self-training, description={The self-training and co-training methods use classifiers first trained on the portion of labelled data to predict the labels of unlabelled instances. The newly labelled samples with the highest confidence are added to the labelled set to retrain the classifiers. The process is repeated iteratively.
}}

%% Privacy
\newglossaryentry{mia}{name=Membership Inference Attack, description={Broadly, an MIA attack aims to determine if a particular record was used to train a machine learning model \cite{chen2019ganleaks}. There is no canonical process by which an attack is conducted, nor specification of the data assets initially in possession of the attacker. For a comprehensive taxonomy of MIA against \gls{gan}, refer to the suitably titled publication by \citeauthor{chen2019ganleaks} in which \gls{medgan} was subjected to a number of trials.}}

%% Training techniques
\newglossaryentry{msn}{
 type=\acronymtype,
 name={MSN},
 description={Per feature, a variational Gaussian mixture model is used to estimate the number of modes and fit a Gaussian mixture. A one-hot vector indicating the mode, and a scalar indicating the value within the mode is produced. See \cite{Xu2019-ay}.},
 text={MSN},
 first={Mode-specific normalization (MSN)}
}

\newglossaryentry{tbs}{
    type=\acronymtype,
    name={TbS}, 
    description={To deal with the imbalance of values in categorical featues, during training the data is resampled in a way that all the categories from discrete attributes are sampled evenly, without inducing bias and so as to recover real data distribution. See \cite{Xu2019-ay} for a step-by-step spefication.}
    text={TbS},
    first={Training by sampling (TbS)}  
}

\newglossaryentry{nnaa}{
    type=\acronymtype,
    name={NN-AA}, 
    description={"Compares the distance from one point in a target distribution T, to the nearest point in a source distribution S, to the distance to the next nearest point in the target distribution." See \cite{yale2019ESANN}.}
    text={NN-AA},
    first={Nearest-neighbor Adversarial Accuracy (NN-AA)}
}

\newglossaryentry{pl}{
    type=\acronymtype,
    name={PL}, 
    description={Difference of \gls{NN-AA} on the test set and on the training set. See \cite{yale2019ESANN}.}
    text={PL},
    first={Privacy loss (PL)}
}

\newglossaryentry{dt}{
    type=\acronymtype,
    name={DT}, 
    description={The discriminator is tested on batches of synthetic data produced by other methods to asses the possibility of overfitting, see \cite{yale2019ESANN}.}
    text={DT},
    first={Discriminator testing (DT)}
}

\newglossaryentry{do}{
    type=\acronymtype,
    name={DO}, 
    description={Privacy preservation method. See \cite{yale2019ESANN} based on \cite{Dwork2008, Prasser2017}.}
    text={DO},
    first={Data obfuscation (DO}
}

\newglossaryentry{pate}{
    type=\acronymtype,
    name={PATE}, 
    description={Differntial privacy method: "The approach combines, in a black-box fashion, multiple models trained with disjoint datasets, such as records from different subsets of users. Because they rely directly on sensitive data, these models are not published, but instead used as "teachers" for a "student" model. The student learns to predict an output chosen by noisy voting among all of the teachers, and cannot directly access an individual teacher or the underlying data or parameters. The student's privacy properties can be understood both intuitively (since no single teacher and thus no single dataset dictates the student's training) and formally, in terms of differential privacy." \cite{Papernot2017,Papernot2018}}
    text={PATE},
    first={Private Aggregation of Teacher Ensembles (PATE)}
}

\newglossaryentry{mbd}{
    type=\acronymtype,
    name={MBD}, 
    description={Training technique. See \cite{Salimans2016}}
    text={MBD},
    first={Mini-batch discrimination (MDB)}
}

\newglossaryentry{t-gan}{
    type=\acronymtype,
    name={T-GAN}, 
    description={Training technique to stabilise training. Allows the introduction of real sample information into the process of training the the generator. See \cite{Jolicoeur-Martineau2019, Su2018}}
    text={T-GAN},
    first={Turing \gls{gan}}
}

\newglossaryentry{corrnn}{
    type=\acronymtype,
    name={CorrNN}, 
    description={Learns a common representation of two views, taking into account their correlation. See \cite{Jolicoeur-Martineau2019, Su2018}}
    text={CorrNN},
    first={Correlation \gls{nn}}
}

\newacronym{}{Grouped CorrNN}{Grouped Correlation Neural Network}

% \newglossaryentry{⟨label ⟩}{type=\acronymtype,
% name={⟨abbrv ⟩},
% description={⟨long⟩},
% text={⟨abbrv ⟩},
% first={⟨long⟩ (⟨abbrv ⟩)},
% plural={⟨abbrv ⟩\glspluralsuffix},
% firstplural={⟨long⟩\glspluralsuffix\space (⟨abbrv ⟩\glspluralsuffix)},
% ⟨key-val list⟩}

%% Algorithms
\newacronym{nn}{NN}{Neural Network}
\newacronym{gan}{GAN}{Generative Adversarial Network}
\newacronym{ohd-gan}{OHD-GAN}{\glspl{gan} for Observation Health Data}
\newacronym{ffn}{FFN}{\gls{feed-forward} Network}
\newacronym{ae}{AE}{Autoencoder}
\newacronym{rnn}{RNN}{Reccurent \gls{nn}}
\newacronym{lstm}{LSTM}{Long Short-term Memory}
\newacronym{cgan}{CGAN}{Conditional \gls{gan}}
\newacronym{crmb}{CRMB}{Conditional Restricted Boltzamann Machine}
\newacronym{cnn}{CNN}{Convolutional \gls{nn}}
\newacronym{wgan}{WGAN}{Wassertein \gls{gan}}
\newacronym{beta-vae}{\ensuremath{\beta}-VAE}{\ensuremath{\beta} variational auto-encoder}
\newacronym{lr}{LR}{Logistic-regression}
\newacronym{cycle-gan}{Cycle-GAN}{Cycle-consistent \gls{gan}}
\newacronym{adtep}{ADTEP}{Adversarial Deep Treatment Effect Prediction}
\newacronym{cae}{CAE}{Convolutional \gls{AE}}

\newacronym[type=oalgo]{medgan}{medGAN}{medGAN}
\newacronym[type=oalgo]{ssl-gan}{SSL-GAN}{Semi-supervised Learning with a learned ehrGAN}
\newacronym[type=oalgo]{wgantpp}{WGANTPP}{\gls{wgan} for Temporal Point-processes}
\newacronym[type=oalgo]{radialgan}{RadialGAN}{RadialGAN}
\newacronym[type=oalgo]{mc-arae}{MC-ARAE}{Multi-categorical gls{arae}}
\newacronym[type=oalgo]{ctgan}{CTGAN}{Conditional Tabular \Gls{gan}}
\newacronym[type=oalgo]{heterogan}{HGAN}{Heterogeneous GAN}
\newacronym[type=oalgo]{emr-wgan}{EMR-WGAN}{EMR Wassertein GAN}
\newacronym[type=oalgo]{corgan}{corGAN}{corGAN}
\newacronym[type=oalgo]{1d-cae}{1D-CAE}{1-dimensional Convolutional \gls{ae}}
\newacronym[type=oalgo]{ehrgan}{ehrGAN}{Electronic Health Record GAN}
\newacronym[type=oalgo]{rgan}{RGAN}{Recurrent \gls{gan}}
\newacronym[type=oalgo]{rcgan}{RC-GAN}{Recurrent Convolutional \gls{gan}}
\newacronym[type=oalgo]{ganite}{GANITE}{Generative Adversarial Nets for inference of Individualized Treatment Effects}
\newacronym[type=oalgo]{cwr-gan}{CWR-GAN}{Cycle Wasserstein Regression \gls{gan}}
\newacronym[type=oalgo]{gain}{GAIN}{Generative Adversarial Imputation Network}
\newacronym[type=oalgo]{mc-medgan}{MC-medGAN}{Multi-categorical \gls{medgan}}
\newacronym[type=oalgo]{mc-gumbelgan}{MC-GumbelGAN}{Multi-categorical Gumbel-softmax \gls{gan}}
\newacronym[type=oalgo]{mc-wgan-gp}{MC-WGAN-GP}{Multi-categorical \gls{wgan} with Gradient Penality}
\newacronym[type=oalgo]{medbgan}{MedBGAN}{Boundry-seeking \gls{medgan}}
\newacronym[type=oalgo]{healthgan}{HealthGAN}{}
\newacronym[type=oalgo]{medwgan}{MedWGAN}{Wassertein \gls{medgan}}
\newacronym[type=oalgo]{sc-gan}{SC-GAN}{Sequentially Coupled \gls{gan}}
\newacronym[type=oalgo]{rmb}{RMB}{Restricted Boltzmann Machine}
\newacronym[type=oalgo]{anomigan}{AnomiGAN}{GANs for anonymizing private medical data}
\newacronym[type=oalgo]{wgan-gp}{WGAN-GP}{\gls{wgan} with Gradient Penality}
\newacronym[type=oalgo]{dp-auto-gan}{DP-auto-GAN}{\gls{DP}-auto-\gls{gan}}
\newacronym[type=oalgo]{ads-gan}{ADS-GAN}{Anonymization through data synthesis using \gls{gan}}
\newacronym[type=oalgo]{gcgan}{GcGAN}{\gls{corrnn} and \gls{t-wgan}}
\newacronym[type=oalgo]{t-wgan}{T-wGAN}{Wassertein \gls{t-gan}}
\newacronym[type=oalgo]{conan}{CONAN}{\textit{Co}plementary patter\textbf{n A}augmentatio\textbf{n } }

%\newacronym[type=oalgo]{}{}{}
%\newacronym{}{}{}

%% Terms
\newacronym{sd}{SD}{Synthetic Data}
\newacronym{ohd}{OHD}{Observational Health Data}
\newacronym{ehr}{EHR}{Electronic Health Record}
\newacronym{icu}{ICU}{Intensive Care Unit}
\newacronym{pmf}{PMF}{Probability Mass Function}


%% Fields
\newacronym[seealso=iteff]{ite}{ITE}{Individual Treatment Effects}
\newacronym[seealso=iteff]{dle}{DLE}{Drug Laboratory Effects}
\newacronym[seealso=pmed]{pm}{PM}{Personalized Medicine}
\newacronym{iot}{IoT}{Internet of Things}

%% Techniques
\newacronym{mba}{MbA}{\gls{mb-avg}}
\newacronym{bn}{BN}{batch-normalization}
\newacronym{sc}{SC}{shortcut connections}
\newacronym{cbt}{CBT}{Cluster-based training}
\newacronym{vcd}{VCD}{Variational contrastive divergence}
\newacronym{ln}{LN}{Layer normalisation}
\newacronym{ssl}{SSL}{Semi-supervised Learning}
\newacronym{sn}{SN}{Spectral Normalization}


%% Privacy
\newacronym{dp}{DP}{Differential privacy}
\newacronym{dp-sgd}{DP-SGD}{Differential private stochastic gradient descent}
\newacronym{ad}{AD}{Attribute Disclosure}
\newacronym[seealso=mia]{pd}{PD}{Presence Disclosure}
\newacronym{rr}{RR}{Reproduction rate}
\newacronym[seealso=mia]{mi}{MI}{Membership Inference}

\newacronym{anm}{ANM}{Additive noise model}


%% Evaluation qualitative
\newacronym{ved}{VED}{Visual Expert Discrimination}

%% Evaluation statistics
\newacronym{dwpro}{DWS}{Dimension-wise Statistics}
\newacronym{dwpre}{DWP}{Dimension-wise Prediction}

%% Evaluation quantitative
\newacronym{fd}{FD}{Feature distributions}
\newacronym{qq}{QQ}{Quantile-quantile plot}
\newacronym{lsr}{LSR}{Latent space representation}
\newacronym{rdp}{RDP}{Renyi Differential Privacy}
\newacronym{pcam}{PCAM}{\gls{pca} Marginal}
\newacronym{pca}{PCA}{Principal Component Analysis}
\newacronym{pcawdd}{PCA-DWD}{\gls{pca} Distributinal Wassertein Distance}

%% Metrics
\newacronym{fop}{F-OP}{First-order proximity}
\newacronym{cc}{CC}{Correlation coefficient}
\newacronym{md-cc}{MD-CC}{\todo{Redundant?}}
\newacronym{mmd}{MMD}{Maximum Mean Discrepency}
\newacronym{rbf}{RBF}{Radial Basis Function}
\newacronym{mse}{MSE}{Mean Squared Error}
\newacronym{auroc}{AUROC}{Area under ROC curve}
\newacronym{auprc}{AUPRC}{Area under the precision-recall curve}
\newacronym{kld}{KLD}{Kullback-Leibler divergence}

%% Evaluation augmentation
\newacronym{tstr}{TSTR}{Train on synthetic, test on real}
\newacronym{trts}{TRTTS}{Train on real, test on synthetic}
\newacronym{pta}{PTA}{Prediction task accuracy}
\newacronym{ssa}{SSA}{Semi-supervised augmentation}






\begin{document}

    

    \title{Synthetic Observational Health Data with GANs: a Cambrian radiation in medical research and digital twins?}
    
    \author[1,2]{Jeremy Georges-Filteau}%
    \author[2]{Elisa Cirillo}%
    \affil[1]{Radboud University Nijmegen}%
    \affil[2]{The Hyve}%


    \vspace{-1em}

    \date{\today}

    \begingroup
    \let\center\flushleft
    \let\endcenter\endflushleft
    \maketitle
    \endgroup

    \selectlanguage{english}

    \glsresetall
    \begin{abstract}
    After being collected for patient care, \gls{ohd} can further benefit patient well-being by sustaining the development of health informatics and medical research. Vast potential is largely unexploited due to the fiercely private nature of patient-related data and regulation about its distribution. \gls{gan} have recently emerged as a groundbreaking approach to efficiently learn generative models that produce realistic \gls{sd}. They have revolutionized practices in multiple domains such as 
     cars, fraud detection, simulations in the industrial sector and marketing known as digital twins, and medical imaging. The digital twin concept could readily be applied to modelling and quantifying disease progression. In addition, \glspl{gan} posses a multitude of capabilities that are directly applicable to common problems in the healthcare: augmenting small dataset, correcting class imbalance, domain translation for rare diseases, let alone preserving privacy. Unlocking open access to privacy-preserving \gls{ohd} could be transformative for scientific research. In the midst of the COVID-19 pandemic, the healthcare system is facing unprecedented challenges, many of which of are data related and cloud be alleviated by the capabilities of \glspl{gan}. In light of these facts, publications concerning the development of  \gls{gan} applied to \gls{ohd} seemed to be severely lacking. To uncover the reasons for the slow adoption of \glspl{gan} for \gls{ohd}, we conducted a broad review of the published literature on the subject. Our findings show that the properties of \gls{ohd} and evaluating the \gls{sd} were initially challenging for the existing \gls{gan} algorithms and metrics (unlike medical imaging, for which state-of-the-art model were directly transferable). Nonetheless, since 2017 solutions to these problems are being published at an increasing rate.
    \end{abstract}

    \glsresetall
\section{Introduction}
    \subsection{Background}
        Medical professionals collect \gls{ohd} in \glspl{ehr} at various points of care in a patient’s trajectory, to support and enable their work \cite{Cowie_2016}. The patient profiles found in \glspl{ehr} are diverse and longitudinal, composed of demographic variables, recordings of diagnoses, conditions, procedures, prescriptions, measurements and lab test results, administrative information, and increasingly omics \cite{Ohdsi2020-vf}.\par
        Having served its primary purpose, this wealth of detailed information can further benefit patient well-being by sustaining medical research and development. That is to say, improving the development life-cycle of \gls{hi}, the predictive accuracy of \gls{ml} algorithms, or enabling discoveries in research on clinical decisions, triage decisions, inter-institution collaboration and \gls{hi} automation \cite{Rudin_2020}. Big health data is the underpinning of two prime objectives of precision medicine: individualization of patient interventions and inferring the workings of biological systems from high-level analysis \cite{Capobianco2020}. However, the private nature of patient-related data, and the growing widespread concern over its disclosure, hampers dramatically the potential for secondary usage of \gls{ohd} for legitimate purposes.\par
        
        Anonymization techniques are used to hinder the misuse of sensitive data. This implies a costly and data-specific cleansing process, and the unavoidable trade-off of enhancing privacy to the detriment of data utility. \todo{ref} These techniques are fallible and do not prevent reidentification. In fact, it has been demonstrated that no polynomial time \gls{dp} algorithms can produce \gls{sd} preserving all relations of the real data, even for simple relations such as 2-way marginals \cite{Ullman2011}. To address these drawbacks, alternative modes for sharing sensitive data is an active research area, including privacy-preserving analytic and distributed learning. Although promising, these approaches come with limitations and their feasibility has yet to be demonstrated. Regardless, distributed models are vulnerable to a variety of attacks, for which no single protection measure is sufficient as research on defense is far behind attack \cite{enthoven2020overview, Gao2020}.The process of \gls{dp} may also \par
        These conditions restrict access to \gls{ohd} to professionals with academic credentials and financial resources. The use of OHD by all other health data-related occupations is blocked, along with the downstream benefits. For example, software developers rarely have access to the data at the core of the \gls{hi} solutions they are developing.
        
    \subsection{Synthetic data}
        An alternative to traditional privacy-preserving methods is to produce full \gls{sd} with methods categorized as either theory-driven (theoretical, mechanistic or iconic) and data-driven (empirical or interpolatory) modelling \cite{Kim_2017, Hand2019}. Theory-driven modelling involves a complex knowledge-based attempt to define a simulation process representing the causal relationships of a system, it's mechanism. The Synthea \cite{Walonoski_2017} synthetic patient generator is one such model, in which state transition models\footnote{Probabilistic model composed of predefined states, transitions, and conditional logic.} produce patient trajectories. The model parameters are taken from aggregate population-level statistics of disease progression and medical knowledge. Such a knowledge-based model depends on prior knowledge of the system, and most importantly how much we can intellect about it \cite{Kim_2017}. On one hand theory-based modelling aims at understanding and offers interpretability, on the other when modelling complex systems, simplifications and assumptions are inevitable, leading to inaccuracies \cite{Hand2019}. In fact, relying on population-level statistics does not produce models capable of reproducing heterogeneous health outcomes \cite{Chen_2019}.\par
        
        Data-driven modelling techniques infer a representation of the data from a sample distribution, in an attempt to summarize or describe it \cite{Hand2019}. There exist numerous statistical modelling approaches to produce \gls{sd}, but the techniques are based on intrinsic assumptions about the data. The representational power is bound to correlations that are intelligible to the modeler, being prone to obscure inaccuracies. \gls{sd} generated by these models tends to hit a ceiling of utility \cite{Rankin2020}. In the ML field, generative models learn an approximation of the multi-modal distribution, from which synthetic samples can be drawn \cite{goodfellow2016nips}. \Gls{gan} \cite{NIPS2014_5423} have recently emerged as a groundbreaking approach to efficiently learn generative models that produce realistic \gls{sd} using \gls{nn}. \gls{gan} algorithms have rapidly found a wide range of applications, such as data augmentation in medical imaging \cite{Yi2019, Wang2020, Zhou2020}.\par
        
        The potential impacts of \gls{gan} to healthcare and science are considerable, some of which have been realized in fields such as medical imaging. However, the application of \gls{gan} to \gls{ohd} seems to have been lagging \cite{Xiao_2018_chall}. Certain characteristics of \gls{ohd} could serve to explain the relatively slow progress. Primarily, algorithms developed for images and text in other fields were easily re-purposed for medical equivalents of the data types. However, \gls{ohd} presents a unique complexity in terms of multi-modality, heterogeneity, and fragmentation \cite{Xiao_2018_chall}. In addition to this, evaluating the realism of synthetic \gls{ohd} is intuitively complex, a problem that still burdens \gls{gan} in general. Nonetheless, in 2017 the first few attempts at \glspl{gan} for \gls{ohd} were published \cite{esteban2017real,Che_2017,Choi2017-nt,yahi2017generative}. We aimed to investigate if the field continued to expand following these first few examples, and if so to gain an comprehensive understanding of methods and approaches to the problem.
    \section{Methods}

    \begin{table}[htb]
  \center
  \footnotesize
    \caption{Search query terms}\label{tab:search}
    \begin{tabular}{@{}clccl@{}} \toprule
	    \multicolumn{2}{c}{Health data} & & \multicolumn{2}{c}{Generative adversarial models} \\ \cmidrule{1-2} \cmidrule{4-5}
	    \multicolumn{2}{c}{Terms} & {} & \multicolumn{2}{c}{Terms} \\ \cmidrule{2-2} \cmidrule{5-5}
	    \multirow{4}{*}{OR} & clinical & \multirow[t]{4}{*}{\quad AND\quad} & \multirow{4}{*}{OR} & generative adversarial\\
	    {} & health & {} & {} & GAN \\ 
	    {} & EHR & {} & {} & adversarial training \\
	    {} & electronic health record & {} & {} & synthetic  \\
	    {} & patient & {} & {} & {} \\
	    \bottomrule
    \end{tabular}
\end{table}
    
    Publications concerning \gls{ohd-gan} were identified through with Google Scholar \cite{scholar}, Web of Science \cite{Clarivate} and Prophy \cite{Prophy}. The search input was formed from the terms and operators found in Table \ref{tab:themes}. We included studies reporting the development, application, performance evaluation and privacy evaluation of \gls{gan} algorithms to produce \gls{ohd}. Broadly, we define the scope of \gls{ohd} as categorical, real-valued, ordinal or binary event data recorded for patient care. A more detailed summary of the included and excluded data types can be found in Table \ref{tab:datatypes}. The data types are already the subject of one or more review, or would merit a review of their own \cite{Yi_2019, Nakata2019, Anwar_2018, Wang2020, Zhou2020}. In each of the included publications, we considered the aspects listed in Table \ref{tab:search}.\par
    
    \begin{table}[htb]
\centering
\footnotesize
  \caption{Aspects analysed in each of the publications included in the review\label{tab:themes}}
  \begin{tabular}{ll}\toprule
  A) Types of healthcare data & D) Evaluation metrics\\
  B) \gls{gan} algorithm, learning procedures, losses & E) Privacy considerations\\
  C) Intended use of the \gls{sd} & F) Interpreatability of the model\\\bottomrule
  \end{tabular}
\end{table}
    
    \begin{table}[htp]
\center
\footnotesize
  \caption{Types of OHD data included or excluded from the review.}\label{tab:datatypes}
  
  \begin{tabularx}{\textwidth}{@{}p{0.1\textwidth}Xp{0.7\textwidth}@{}}\toprule
  Type & Examples \\ \midrule
  
  \multirow{4}{*}{Included} & Observations & Demographic information, medical classification, family history \\
  &Timestamped observations & Diagnosis, treatment and procedure codes, prescription and dosage, laboratory test results, physiologic measurements and intake events \\
  &Encounters & Visit dates, care provider, care site \\
  &Derived & Aggregated counts, calculated indicators \\ \midrule

  \multirow{4}{*}{Excluded} &Omics & Genome, transcriptome, proteome, immunome, metabolome, microbiome \\
  &Imaging & X-rays, computed tomography (CT), magnetic resonance imaging (MRI) \\
  &Signal & Electrocardiogram (ECG), electroencephalogram (EEG) \\
  &Unstructured & Narrative reports, textual \\ \bottomrule
  \end{tabularx}%
\end{table}








    \section{Results}
    \subsection{Summary}
        We have found a total of 43 \todo{recount and check for more} publications describing the development or adaption of \gls{ohd-gan}, presented in Table \ref{tab:3:publications}. The type of data addressed in each of these publications can be generalized into one of two categories: time-dependent observations, such as time-series, or static representation in the form of feature vectors such as tabular rows.\par
        
        Most efforts propose adaptations of current algorithms to the characteristics and complexities of \gls{ohd}. These include multi-modality of marginal distributions or non-Gaussian continuous features, heterogeneity, a combination of discrete and continuous features, longitudinal irregularity, complex conditional distributions, missingness or sparsity, class imbalance of categorical features and noise.\par 
        
        While these properties may make training a useful model difficult, the variety of applications that are highly relevant and needed in the healthcare domain provide sufficient incentive. The most cited motives are, as one would expect, to cope with the often limited number of samples in medical datasets and to overcome the highly restricted access to \gls{ohd}. The potential of releasing privacy-preserving \gls{sd} freely is a common subject. Publications considering privacy evaluate the effect on utility of applying \gls{dp} to their algorithm, propose alternatives privacy concepts and metrics, or exclusively concentrate on the subject of privacy.\par
        
    \subsection{Motives for developing OHD-GAN}
        Some claim that the ability to generate synthetic is becoming an essential skill in data science \cite{Sarkar2018}, but what purpose can it serve in the medical domain? The authors mention a wide range of potential applications. We briefly describe the four prevailing themes in the following sections: data augmentation (Sec.\ref{sec:augmentation}), privacy and accessibility (Sec.\ref{sec:access_privacy}), precision medicine (Sec.\ref{sec:precision_med}) and  modelling simulations (Sec.\ref{sec:models_twins}). 

        \subsubsection{Data augmentation}\label{sec:augmentation}
    
            Data augmentation \todo{define} is mentioned in nearly all publications. Although counter-intuitive, it is well known that \gls{gan} can generate \gls{sd} that conveys more information about the real data distribution. Effectively, the continuous space distribution of the generator produces a more comprehensive set of data points, valid, but not present in the discrete real data points. A combination of real and synthetic training data habitually leads to increased predictor performance \cite{Wang_2019,Che_2017,Yoon2018-ite, yoon2018imputation, Yang_2019_impute_ehr, Chen_2019, cui2019conan, Che_2017}. A more intelligible way to seize the concept from the point of view of image classification, in which it is known as invariances, perturbations such as rotation, shift, sheer and scale \cite{antoniou2017data}.\par 
            
            Similarly, domain translation \todo{define} and \gls{semi-sup} training approaches with \glspl{gan} could support predictive tasks that lack data with accurate labels, lack paired samples or suffer class imbalance \cite{Che_2017,mcdermott2018semi, Yoon2018-ite}. Another example is correcting discrepancies between datasets collected in different locations or under different conditions inducing bias \cite{Yoon2018-radial}. \glspl{gan} are also well adapted for data imputation, were  entries are \gls{mar} \cite{yoon2018imputation}. 

        \subsubsection{Enhancing privacy and increasing data accessibility}\label{sec:access_privacy}
    
            \gls{sd} is seen as the key to unlocking the unexploited value of \gls{ohd} hindering machine learning, and more generally scientific progress \cite{Beaulieu-Jones2019-ct, baowaly_2019_IEEE,baowaly_2019_jamia,Che_2017,esteban2017real,Fisher2019,severo2019ward2icu}. Preserving privacy can broadly be described as reducing the risk of \gls{re-iden} to an acceptable level. This level of risk is quantified when releasing data anonymized with \gls{dp}.\par
    
            Due to its artificial nature, \gls{sd} is put forward as a means to forgo the tight restrictions on data sharing, while potentially providing greater privacy guarantees \cite{Beaulieu-Jones2019-ct, baowaly_2019_IEEE, baowaly_2019_jamia,esteban2017real,Fisher2019,walsh2020generating, chin2019generation}. Enabling access to greater variety, quality and quantity of \gls{ohd} could have positive effects in a wide range of fields, such as software development, education, and training of medical professionals. The fact remains that \glspl{gan} do not eliminate the risk of reidentification. Considering none of the synthetic data points represent real people, the significance of such an occurrence is unclear. \todo{find some more information} Nonetheless, both methods can be combined, and \gls{gan} training according to \gls{dp} shows evidence of reducing the loss of utility in comparison to \gls{dp} alone. \todo{Find these citations} Overall,  
    
        \subsubsection{Enabling precision medicine}\label{sec:precision_med}
    
            The application to precision medicine generally involve predicting outcomes conditioned on a patient's current state and history. Simulated trajectories could help inform clinical decision making by quantifying disease progression and outcomes and have a transformative effect on healthcare \cite{walsh2020generating, Fisher2019}. Ensembles of stochastic simulations of individual patient profiles such as those produced by \gls{crmb} could help quantify risk at an unprecedented level of granularity \cite{Fisher2019}.\par
            Predicting patient-specific responses to drugs is still a new field of research, a problem known as \gls{ite}. The task of estimating \gls{ite} is persistently hampered by the lack of paired counterfactual samples \cite{Yoon2018-ite, chu2019treatment}. To solve similar problems n medical imaging, various \gls{gan} algorithms were developed for domain translation, mapping a sample from its to original class to the paired equivalent. This includes bidirectional transformations, allowing \gls{gan} to learn mappings from very few, or a lack of paired samples \cite{Wolterink2017DeepMT, CycleGAN2017, mcdermott2018semi}.
    
        \subsubsection{From patient and disease models to digital twins}\label{sec:models_twins}
    
            A well trained model approximates the process that generated the real data points. In other words, the relations learned by the model, its parameters, contains meaningful information if we can learn to harness it. Data-driven algorithms evolve as our understanding of their behavior improves. New concepts are incorporated in the algorithms leading to further understanding, iterativly blurring the line with theory-driven approaches \cite{Hand2019}. Interpretability is a growing field of research concerned with understanding how the learned parameters of a model relate. In other words analysing the representation the algorithm has converged to and deriving meaning from seemingly obscure logic.Incorporating new understanding in the architecture of algorithms shift the view from a data-driven to a theory-driven perspective \cite{Hand2019}. As we purposefully build structure in our algorithms from new understanding we may get the chance to explore meaningful representations that would otherwise be beyond our reasoning.\par 
            
            Approaching these ideas from above, the concept of "digital twins" represents in a way the ultimate realization of \gls{pm}. A common practice in industrial sectors is high-fidelity virtual representations of physical assets. Long-term simulations, that provide an overview and comprehensive understanding of the workings, behavior and life-cycle of their real counterparts. The state of the models is continuously updated from theoretical data, real data and streaming \gls{iot} indicators.\par
            Intently conditioned input data allows the exploration of specific events or conditions. In a position paper on the subject, Angulo et al. draw the parallels of this technique with the current needs in healthcare and the emergence of the necessary technologies for actionable models of patients. \cite{angulo2019towards,Angulo_2020}. The authors bring up the rapid adoption of wearables that are continuously monitoring people's physiological state. 
            Wearables are one of many mobile digitally connected devices that collect patient data over a broad range of physiological characteristic and behavioral patterns \cite{coravos2019developing}. This emerging trend known as \gls{dbio} has already led to studies demonstrating predictive models with the potential for improved patient care \cite{snyder2018best}. Through continuous lifelong learning, integrating  multiple modes of personal data, generative patient models could inform diagnostics of medical professionals and also enable testing treatment options. In their proposal, \gls{gan} are an essential component of the ecosystem to ensure patient privacy and to provide bootstrap data. Fisher et al. already employ the term "digital twin" to describe their process, noting that they present no privacy risk and enable simulating patient cohorts of any size and characteristics \cite{walsh2020generating}.
        
            
\newcolumntype{R}{>{\raggedright\arraybackslash}p{0.20\textwidth}}
\newcolumntype{M}{>{\raggedright\arraybackslash}p{0.40\textwidth}}
\newcolumntype{N}{>{\raggedright\arraybackslash}p{0.30\textwidth}}
\newcommand{\specialcell}[2][c]{%
  \begin{tabular}[#1]{@{}l@{}}#2\end{tabular}}
  
\begin{center}
    
    \setlength\LTleft{0pt}
    \setlength\LTright{0pt}
    \scriptsize
    \setlength{\extrarowheight}{1em}
    
    \begin{longtable}[l]{@{}p{0.10\textwidth}NMR@{}} 
        \kill
        \caption{Summary of the publication included in the review.\label{tab:3:publications}}\\
        \hline
        Publication & Algorithm(s) & Focus, algorithms and techniques & Data type \\ 
        \hline
        \endfirsthead
        \caption[]{Summary of the publication included in the review (Continued).}\\
        \hline
        Publication & Algorithm(s) & Focus, algorithms and techniques & Data \\ 
        \hline
        \endhead
        \hline 
        \endfoot
        
        \quad 2017 & & & \\
        \hline
        \citeauthor{Choi2017-nt} & \gls{medgan} 
        & Incompatibility of back-propagation with discrete features. \gls{ae}, \gls{mb-avg}, \gls{bn}, \gls{sc}, \gls{ad}, \gls{pda}.
        & Binary occurences or counts of medical codes.\\
        
        \citeauthor{yahi2017generative} & \gls{medgan} adaptation
        & \Gls{dle} on continuous time-series, multi-modality. \gls{t-sne}.
        & Paired pre/post treatment exposure time-series\\
        
        \citeauthor{esteban2017real} & \gls{rgan}, \gls{rcgan} 
        &  Adversarial training of (conditional) \glspl{rnn} on time-series, evaluation, privacy. \gls{lstm}, \gls{cgan}, \gls{dp-sgd}.
        & Regularly observed \gls{rvts}\\
        
        \citeauthor{Xiao2017-lh} & \gls{wgantpp} 
        & Temporal Point Processes. \gls{lstm}, \gls{wgan}, Poisson process.
        & Sporadic occurrences, hospital visits.\\
        
        \citeauthor{Che_2017} & \gls{ehrgan}, \gls{ssl-gan} 
        & Semi-supervised augmentation, transitional distribution. 1D-CNN, Word2vec, \gls{vcd}.
        & \Gls{dts}, sequences of medical codes. \\
        
        \citeauthor{dash2019synthetic} & \gls{healthgan} & Sleep patterns, stratification by covariates. & Binary over multiple visits.\\
        
        \hline
        \quad 2018 & & & \\
        \hline
        
        \citeauthor{Camino2018-re} & \raggedright \gls{mc-arae}, \gls{mc-medgan}, \gls{mc-gumbelgan}, \gls{mc-wgan-gp}
        & Improving training process. \gls{medgan}, \gls{wgan-gp}, \gls{gumbel-gan}, \gls{arae}.
        & Multiple categorical variables. \\
        
        \citeauthor{mcdermott2018semi} & \gls{cwr-gan}
        & Cycle-consistent semi-supervised regression learning, unpaired data, class imbalance. \gls{wgan} \gls{cycle-gan} \gls{ite}
        & ICU \gls{rvts}, lack of paired samples, \gls{sd}. \\
        
        \citeauthor{Yoon2018-ite} & \gls{ganite} 
        & \gls{ite}, unobserved counterfactual, multi-label classification, uncertainty. \gls{cgan} pair.
        & Feature, treatment and outcome vectors.\\
        
        \citeauthor{Yoon2018-radial} & \gls{radialgan} 
        & Multi-domain translation, features and distribution mismatch, cycle-consistency, augmentation. \gls{cgan}, \gls{wgan}.
        & Tabular, discrete and continuous.\\
        
        \citeauthor{yoon2018imputation} & \gls{gain}
        & Tabular data imputation. \gls{mcar}, \gls{cgan}.
        & Real-valued, tabular with entries \gls{mcar}.\\
        
        \hline
        \quad 2019 & & & \\
        \hline
        
        \citeauthor{Wang_2019} & \gls{sc-gan}
        & Capturing mutual influence in time-series. Coupled generator pair. Treatment recommendation task. \gls{lstm}, \gls{cgan}.
        & \Gls{rvts} of patient state and medication dosage data.\\
        
        \citeauthor{baowaly_2019_IEEE} & \gls{medbgan}
        & Improving training process. \gls{medgan}, \gls{bgan}.
        & Binary occurences or counts of medical codes.\\
        
        \citeauthor{baowaly_2019_jamia} & \gls{medbgan}, \gls{medwgan}
        & Improving training process. \gls{medgan}, \gls{bgan}, \gls{wgan}.
        & Binary occurences or counts of medical codes.\\
        
        \citeauthor{severo2019ward2icu} & \gls{cwgan-gp} 
        & Generation and public release of dataset. Protecting commercial sensitive information. Class imbalance. \gls{cwgan-gp}, \gls{cgan}.
        & Physiological \gls{rvts}.\\
        
        \citeauthor{chin2019generation} & \gls{wgan}
        & Heterogeneous mixture of dense and sparse features. Privacy and evaluating the introduction of bias. \gls{wgan}, \gls{wgan-gp}, \gls{msn}, \gls{dp} aware optimizer from Tensor-flow. \citeauthor{tensorflow-privacy}.
        & Binary, real-valued and categorical.\\
        
        \citeauthor{Jordon2019} & \gls{pate-gan}
        & Alternative differential privacy, adaptation of the  \gls{pate} framework.
        & Demographic and binary.\\
        
        \citeauthor{torfi2019generating} & \gls{corgan}
        & \gls{cnn} architecture, capturing feature correlations, evaluating realism, privacy evaluation using \gls{mi}. 1D-\gls{cae}.
        & Binary occurences or counts of medical codes.\\
    
        \citeauthor{chu2019treatment} & \gls{adtep}
        & \gls{ite}, two independent \gls{ae} for patient and treatment feature sets, trained adversarially in combination, and outcome predictor from latent representation. &
         \Gls{ehr} data, not specified.\\
        
        \citeauthor{Jackson_2019} & \gls{medgan}
        & Evaluating medgan with the addition of demographics features.
        & Demographic features and binary occurences or counts of medical codes. \\
        
        \citeauthor{yu2019rare} & \gls{ssl-gan}
        & Rare disease detection, \gls{ssl}, leveraging unlabeled \gls{ehr} data, medical code embedding network.  \gls{lstm}.
        & Diagnosis and prescription codes.\\
        
        \citeauthor{Yang_2019_cdss} & \gls{cgan}
        & Class imbalance, low count of minority class. Semi-supervised learning combining \gls{st} and \gls{ct} with a \gls{cgan} for a \gls{iot} application.
        & Twenty medical datasets from the UCI repository, types unspecified.\\
        
        \citeauthor{Yang_2019_ehr} & \gls{gcgan}
        & Capturing the correlations between different categories of medical codes and the outcome.  \gls{corrnn}, \gls{t-gan}, \gls{t-wgan}.
        & Binary occurences or counts of medical codes.\\
        
        \citeauthor{Yang_2019_impute_ehr} & \gls{cgain}
        & Improve on \gls{gain} for categorical variable using fuzzy encoding of the features. 
        & Categorical (multi-class and multi-label) real-valued.\\
        
        \citeauthor{Camino2019} & \gls{gain}, \gls{gain}+\gls{vs}, \gls{vae}, \gls{vae}+\gls{it}, \gls{vae}+\gls{bp},  \gls{vae}+\gls{vs},  \gls{vae}+\gls{vs}+\gls{it}, \gls{vae}+\gls{vs}+\gls{bp}
        & Benchmark and improve on generative imputation with \gls{gain} and \gls{vae}. & Categorical and real-valued. Mostly not \gls{ohd}\\
        
        \citeauthor{Beaulieu-Jones2019-ct} & \gls{ac-gan} 
        & Evaluating if differentially private GANs that is valid reanalysis while ensuring privacy.  \gls{dp}, \gls{cgan}.
        & Physiological \gls{rvts}.\\
        
        \citeauthor{Xu2019-ay} & \gls{ctgan}
        & Non-Gaussian multi-modal distribution of continuous columns and imbalanced discrete column in tabular data. Evaluation benchmark.  \gls{cgan} \gls{tbs} \gls{msn} \gls{wgan-gp} \gls{gumbel-gan}
        & Tabular real-valued and categorical.\\
        
        \citeauthor{yale2019ESANN} & \gls{healthgan}
        & Privacy metrics and over-fitting.  \gls{mi}, \gls{nnaa}, \gls{pl}, \gls{dt}
        & Categorical demographics, real-valued and binary medical codes.\\
        
        \citeauthor{Fisher2019} & Adversarially trained \gls{crmb}
        & Simulation of patient trajectories from their baseline state, disease prediction and risk quantification, missingness.\gls{crmb}.
        &  Binary, ordinal, categorical, and continuous, 3 months intervals.\\
        
        \hline
        \quad 2020 & & & \\
        \hline
        
        \citeauthor{walsh2020generating} & Adversarially trained \gls{crmb}
        & Digital twins, disease prediction and risk quantification, missingness.   \gls{crmb}.
        & Binary, ordinal, categorical, and continuous, 3 months intervals.\\
        
        \citeauthor{Yale_2020} & \gls{healthgan}
        & Metrics to capture a synthetic dataset’s resemblance, privacy, utility and footprint. Evaluating applications. Application case studies, Reproducibility of studies with \gls{sd}.  \gls{nnaa}, \gls{pl}, \gls{do}, \gls{medgan}, \gls{wgan-gp}, \gls{sdv}, 
        & Real-valued and categorical. Demographics, vital signs, diagnoses, and procedures.\\
        
        \citeauthor{tanti2019} & \gls{dp-auto-gan}
        & Privacy, \gls{medgan} adaptation, evaluation metrics.  \gls{dp-sgd} \gls{ae} \gls{medgan} \gls{rdp}
        & Medical data: binary. Non-health data: categorical and real-valued.\\
        
        \citeauthor{BaeAnomiGAN2020} & \gls{anomigan}
        & Probabilistic scheme that ensures \textit{indistinguishability} of the \gls{sd}, than can be viewed as encrypted.  \gls{dp} \gls{cnn}
        & Binary occurences of medical codes.\\
        
        \citeauthor{cui2019conan} & \gls{conan}
        & Complementary \gls{gan} in a rare disease predictor model that generates positive samples from negatives to alleviate class imbalance.
        & Embedding vectors representing multiple patient visits and conditions.\\
        
        \citeauthor{zhu_2020} & \gls{glugan}
        & Adversarialy trained \gls{rnn} to predict the upcoming time-step in physiological time-series conditioned on the past observations.  \gls{rnn}, \gls{cnn}, \gls{gru}.
        & \Gls{rvts} of blood glucose measurements, discrete patient submitted features.\\
        
        \citeauthor{chen2019ganleaks} & \gls{medgan}, \gls{wgan-gp}, DC-GAN
        & Privacy analysis of generative models.   \gls{mi}, \gls{fullbba}, \gls{partbba}, \gls{wba}, \gls{dp-sgd}.
        & Binary vector of medical codes.\\
        
        \citeauthor{chincheong2020generation} & \gls{wgan-dp}
        & Heterogeneous data, effect of differential privacy on utility.   \gls{wgan} \gls{dp}
        & Categorical, continuous,  ordinal, and binary. Dense or sparse.\\
        
        \citeauthor{Camino2020bench} Initially a comparison \glspl{gan} and \glspl{vae}, but they choose instead to bring attention to the problem of benchmarking. Analysis of problematic,  requirements and suggestions. & Real-valued and categorical.
        
        \citeauthor{Zhang2020} & \gls{emr-wgan}
        & Improving training, evaluation metrics, sparsity.  \gls{wgan}, \gls{bn}, \gls{ln}, \gls{cgan}.
        & Binary occurences of medical codes. Low-prevalence of codes. \\
        
        \citeauthor{yan2020generating} & \gls{heterogan}
        & Improvements on \gls{emr-wgan} incorporating record-level constraints in the loss function.   \gls{wgan}, \gls{bn}, \gls{ln}, \gls{cgan}, \gls{mi}, \gls{pda}.
        & Binary, categorical and real-valued.\\
        
        \citeauthor{ozyigit2020generation} & \gls{rsdgm}
        & Exploring the feasibility of various methods to generate synthetic datasets.   \gls{wgan}
        & Real-valued and categorical.\\
        
        \citeauthor{Yoon2020-anon} & \gls{ads-gan}
        & Identifiability view of privacy. Generator conditioned on real samples inputs with an identifiability loss to satisfy the identifiability constraint.   \gls{wgan} \gls{wgan-gp} \gls{dp} alternative.
        & Real-valued and binary.\\
        
        \citeauthor{Goncalves2020} & \gls{mc-medgan}
        & Comparison of \glspl{gan} with statistical models to generate synthetic data, evaluation metrics.   \gls{mi}, \gls{ad}.
        & Categorical and real-valued.\\
           
        \hline
        
    \end{longtable}
\end{center}

    \subsection{Data Types and Feature Engineering}

        No publications made use of \gls{ohd} in its initial form, patient records in \gls{ehr} composed of many related tables (normalized form). The complexity of a model would grow rapidly when maintaining referential integrity and statistics between multiple tables. The hierarchy by witch these would interact with each other conditionally is no less complicated \cite{Buda2015, Patki_2016, Zhang2015, Tay2013}. There are however published \gls{gan} algorithms made to consume normalized database in their original form. \todo In regards to \gls{ohd}, feature engineering was used to adapt the data to task requirements, or to a promising algorithms that fit the date characteristics. The data is transformed into one of four modalities: time series, point-processes, ordered sequences or aggregates described in Fig. \ref{tab:features}.

        \begin{table}[H]
        \footnotesize
        \caption{Types of observational health data and features engineering}\label{tab:features}
    
        \begin{tabularx}{\textwidth}{@{}p{0.15\textwidth}p{0.3\textwidth}p{0.3\textwidth}X@{}} \toprule
            Type & Values and structure & Challenges & Features engineering\\ \midrule
            
            \textbf{Time-series}\newline
            \textit{Continuous}\newline 
            \textit{Regular}\newline
            \textit{Sporadic}
            &\begin{minipage}[t]{0.3\textwidth}{
            \begin{itemize}[leftmargin=*]  
                \item Timestamped observations 
                \item Continuous, ordinal, categorical and/or multi-categorical
                \item Recorded continuously by medical devices, following a schedule by medical professional, or when necessary
            \end{itemize}}
            \end{minipage}
            &\begin{minipage}[t]{0.3\textwidth}{
            \begin{itemize}[leftmargin=*]  
                \item Observations are often \gls{mar} across time end dimensions, erroneous, or completely absent for certain patients.
                \item Time-series of different concepts are often highly correlated and their influence on one another must be accounted for.
             \end{itemize}}
             \end{minipage}
            & Imputation coupled with training \newline Regular \newline Data imputation \newline Binning in into fixed-size intervals \newline Combination of binning and imputation \\
            
            \textbf{Point-processes} 
            &\begin{minipage}[t]{0.3\textwidth}{
            \begin{itemize}[leftmargin=*]  
                \item Series of timestamped observations of one variable or medical concept per patient
            \end{itemize}}
            \end{minipage}
            &\begin{minipage}[t]{0.3\textwidth}{
            \begin{itemize}[leftmargin=*]  
                \item \todo{Intensity functions, paramtric models}
            \end{itemize}}
            \end{minipage}
            & Series of events reduced to the time interval between each consecutive occurrence. \\ 
            
            \textbf{Ordered \linebreak sequences} 
            & \begin{minipage}[t]{0.3\textwidth}{
            \begin{itemize}[leftmargin=*]  
                \item Ordered vectors representing one or more patients visits
                \item Medical codes associated with the diagnoses, procedures, measurements and interventions
            \end{itemize}}
            \end{minipage}
            & Variable length\newline High-dimensional\todo\newline Long-tail distribution of codes 
            & Sequences are projected into a trained embedding that preserves semantic meaning according to methods borrowed from NLP\\
            
            \textbf{Tabular}\newline Denormalized\newline Relational
            &\begin{minipage}[t]{0.3\textwidth}{
            \begin{itemize}[leftmargin=*]  
                \item Medical and demographic variables aggregated in tabular format
                \item Continuous, ordinal, categorical and/or multi-categorical features
            \end{itemize}}
            \end{minipage}
            & Medical history is aggregated into a fixed-size vector of binary or aggregated counts of occurrences and combined with demographic features.\\
            
            \bottomrule
        \end{tabularx}
    \end{table}

    \subsection{Data oriented GAN development}\label{subsec:data_gan_dev}

        \subsubsection{Auto-encoders and categorical features}\label{subsubsec:categorical}

            In what is to the best of our knowledge, the first attempt at developing a \gls{gan} for OHD. \citeauthor{Choi2017-nt} focus on the problem posed by the incompatibility of categorical and ordinal features with back-propagation. Their solution is to pretrain an \gls{ae} to project the samples to and from a continuous latent space representation. The decoder portion is retained along with its trained weights to form a component of \gls{medgan} \cite{Choi2017-nt}. It is incorporated into the generator and maps the randomly sampled input vectors from the real-valued latent space representation back to discrete features. This first exemplar of synthetic OHD generated by \gls{gan} inspires a series of enhancements.\par

            Numerous efforts were made to improve the performance of \gls{medgan}. Among the first, \citeauthor{Camino2018-re} developed \gls{mc-medgan} in which they modified the \gls{ae} component by splitting its output into a Gumbel-Softmax \cite{jang2016categorical} activation layer for each categorical variable and concatenating the results. \cite{Camino2018-re}. The authors also developed an adaptation based on recent training techniques: \gls{wgan} \cite{arjovsky2017wasserstein} and a \gls{wgan} with Gradient Penalty \cite{gulrajani2017improved}. In brief, the Wasserstein distance is a measure between two \glspl{pd} that has the property of always providing a smooth gradient. When used as the loss function of the discriminator, it generally improves training stability and mitigates mode collapse. The Wassertein loss function a 1-Lipschitz constraint that was originally solved by weight clipping. It was however demonstrated that in some cases this prevented the network from modelling the optimal function, thus Gradient penalty, a less restrictive regularization was introduced \cite{Petzka2018}. \gls{mc-wgan-gp} is the equivalent of \gls{mc-medgan} but with Softmax layers. The authors report that the choice of a model will depend on data characteristics, particularly sparsity.\par
            
            Wasserstein's distance was widely adopted by subsequent authors owing to the propensity of OHD to induce mode collapse. Baowaly et al. developed \gls{medwgan} also based on \gls{wgan}, and \gls{medbgan} borrowing from Boundary-seeking \gls{gan} (BGAN) \cite{hjelm2017boundaryseeking} which pushes the generator to produce samples that lie on the decision boundary of the discriminator, expanding the search space. Both led to improved data quality, in particular \gls{medbgan} \cite{baowaly_2019_IEEE,baowaly_2019_jamia}. In other effort, \citeauthor{Jackson_2019} tested \gls{medgan} on an extended dataset containing demographic and health system usage information, obtaining results similar to the original \cite{Jackson_2019}. The \gls{healthgan} built upon \gls{wgan-gp}, but includes a data transformation method adapted from the Synthetic Data Vault \cite{Patki_2016} to map categorical features to and from the unit numerical range \cite{Yale_2020}. 
        
        \subsubsection{Forgoing the autoencoder and conditional training}\label{noauto}

            Claiming that the use of an \gls{ae} introduces noise, with \gls{emr-wgan}, \citeauthor{Zhang2020} dispose of the \gls{ae} component of previous algorithms and introduce a conditional training method, along with conditioned \gls{bn} and \gls{ln} techniques to stabilise training \cite{Zhang2020}. The algorithm was further adapted by \citeauthor{yan2020generating} as \gls{heterogan} to better account for the conditional distributions between multiple data types and enforce record-wise consistency. A recognized problem with \gls{medgan} was that it produced common-sense inconsistencies, such as gender mismatches in medical codes \cite{yan2020generating, Choi2017-nt}. In \gls{heterogan}, constraints are enforced by adding specific penalties to the loss function, such as limit ranges for numerical categorical pairs and mutual exclusivity for pairs of binary features \cite{yan2020generating}. The algorithm also performs well on regular time-series of sleep patterns \cite{dash2019synthetic} \par

            To develop \gls{ctgan}, \citeauthor{Xu2019-ay} presume that tabular data poses a challenge to \gls{gan} owing to the non-Gaussian multi-modal distribution of continuous columns and imbalanced discrete columns \cite{Xu2019-ay}. Their algorithm, composed of fully connected layers, was developed with adaptations to deal with both continuous and categorical features. For continuous features, it employs \todo{define} mode-specific normalization  to capture the multiplicity of modes. For discrete features conditional \todo{define} training-by sampling is devised to re-sample discrete attributes evenly during training, while recovering the real distribution when generating data.\par
            
            In other efforts, \citeauthor{torfi2019generating} develop \gls{corgan}, in which the \gls{ae} is replaced by a \gls{1d-cae} to capture neighboring feature correlations of the input vectors \cite{torfi2019generating}. \citeauthor{chincheong2020generation} use a \gls{ffn} based on Wassertein distance to evaluate the capacity of \glspl{gan} to model heterogeneous data of dense and sparse medical features \cite{chincheong2020generation}. \citeauthor{ozyigit2020generation} use the same approach with regards to reproducing statistical properties \cite{ozyigit2020generation}.
            
        \subsubsection{Time-series}
            
             Reproducing physiological time-series \citeauthor{esteban2017real} devise the \gls{rgan} and \gls{rcgan} based on \gls{lstm} to generate a regular time-series of physiological measurements from bedside monitors \cite{esteban2017real}. Curiously, the authors dismiss Wassertein's distance, stating that they did not find application in their experiments. In addition, each dimension of their time-series is generated independently from the others, where one would assume they are correlated. A considerable loss of accuracy is observed on their utility metric. In a benchmark on non-health data with long-term dependencies and complex multidimensional relationships against DopelGANger it was outperformed \cite{Lin2019}. \todo{Move to discussion}

    \subsection{Task oriented GAN development}
        \subsubsection{Semi-supervised learning}

            To develop \gls{ehrgan}, an algorithm for sequences of medical codes that learns a transitional distribution, \citeauthor{Che_2017} combine an Encoder-Decoder \gls{cnn} \cite{Rankin2020} with \gls{vcd} \cite{Che_2017}. The \gls{ehrgan} generator is trained to decode a random vector mixed with the latent space representation of a real patient. The trained \gls{ehrgan} model is then incorporated into the loss function of a predictor where it can help generalization by producing neighbors for each input sample.\par
            
            \Gls{ssl} is commonly employed to augment the minority class in imbalanced datasets, techniques such as \gls{st} and \gls{ct}. \citeauthor{yang2018unpaired} improve on both of these by incorporating a \gls{gan} in the procedure \cite{yang2018unpaired}. The \gls{gan} is first trained on the labelled set and used to re-balance it. A prediction task with a classifier ensemble is then executed and the data points with highest prediction confidence are labelled. The process is iterated until labelling expansion ceases. As a final step, the \gls{gan} is trained on the expanded labelled set to generate an equal amount of augmentation data. The authors obtained improved performance in a number of classification tasks and multiple tabular datasets with their method.
    
    \subsubsection{Domain translation}
    
        To address the heterogeneity of healthcare data originating from different sources, \citeauthor{Yoon2018-radial} combines the concepts of cycle-consistent domain translation \todo{define} from \gls{cycle-gan} \cite{Zhu_2017} and  multi-domain translation from Star-GAN \cite{choi2017stargan} to build \gls{radialgan} to translate heterogeneous patient information from different hospitals, correcting features and distribution mismatches \cite{Yoon2018-radial}. The algorithm use an encoder-decoder pair per data endpoint that are trained to map records to and from a shared latent representation for their respective endpoint. 
    
    \subsubsection{Individualized treatment effects}
    
        The task of estimating \glspl{ite} is an ongoing problem. \glspl{ite} refer to the response of a patient to a certain treatment given a set of characterizing features. This is due to the fact that counterfactual outcomes are never observed or treatment selection is highly biased \cite{Yoon2018-ite, mcdermott2018semi, walsh2020generating}. To overcome this problem, in \gls{ganite} \citeauthor{Yoon2018-ite} employ a pair of \glspl{gan}, one for counterfactual imputation and another for \gls{ite} estimation \cite{Yoon2018-ite}. The former captures the uncertainty in unobserved outcomes by generating a variety of counterfactuals. The output is fed to the latter, which estimates treatment effects and provides confidence intervals.\par
    
        With \gls{cwr-gan}, a joint regression-adversarial model, \citeauthor{mcdermott2018semi} demonstrated a \gls{ssl} approach inspired by \gls{cycle-gan} to leverage large amounts of unpaired pre/post-treatment time-series in \gls{icu} data for the estimation of \glspl{ite} on physiological time-series \cite{mcdermott2018semi}. The algorithm has the ability to learn from unpaired samples, with very few paired samples, to reversibly translate the pre/post-treatment physiological series.\par 
    
        \citeauthor{chu2019treatment} approach the problem of data scarcity by designing \gls{adtep}, an algorithm that can maximize use of the large volume of \gls{ehr} data formed by triples of non-task specific patient features, treatment interventions and treatment outcomes \cite{chu2019treatment}. \gls{adtep} learns representation and discriminatory features of the patient, and treatment data by training an \gls{ae} for each pair of features. In addition to \gls{ae} reconstruction loss, a second model is tasked with advsersarially identifying fake treatment feature reconstructions. Finally, a fourth loss metric is calculated by feeding the concatenated latent representations of both \glspl{ae} to a \gls{lr} model aimed at predicting the treatment outcome \cite{chu2019treatment}.\par
    
        Similarly to \citeauthor{esteban2017real}, \citeauthor{Wang_2019} demonstrated an algorithm to generate a time series of patient states and medication dosages pairs using \gls{lstm}. In contrast to \gls{rgan} and \gls{rcgan}, in \gls{sc-gan}, patients state at the current time-step informs the concurrent medication dosage, which in turn affects the patient state in the upcoming time-step \cite{Wang_2019}. \gls{sc-gan} overcame a number of baselines on both statistical and utility metrics.
    
    \subsubsection{Data imputation and augmentation}
    
        \gls{gan} are naturally suited for data imputation, and could provide a new approach to deal with the problems of health data relating to widespread missingness. Statistical models developed for the multiple imputation problem increase quadratically in complexity with the number of features, while the expressiveness of deep neural networks can efficiently model all features with missing values simultaneously.In that regard, \citeauthor{yoon2018imputation} adapted the standard \gls{gan} to perform imputation on continuous features \gls{mar} in tabular datasets \cite{yoon2018imputation}. In \gls{gain}, the discriminator is tasked with classifying individual variables as real or fake (imputed), as opposed to the whole ensemble. Additional input, or hint, containing the probability of each component being real or imputed is fed to the discriminator to resolve the multiplicity of optimal distributions that the generator could reproduce. The model performs considerably better than five state-of-the-art benchmarks. \gls{gain} was later adapted by \citeauthor{Yang_2019_impute_ehr} to also handle categorical features using fuzzy binary encoding, the same technique employed in \gls{healthgan}. In parallel, \citeauthor{Camino2019} apply the same \gls{vs} technique they used fir \gls{medgan} to adapt \gls{gain} and run a benchmark against different types of \gls{vae}.\par
    
        The distribution estimated by a generator model can compensate for lack of diversity in a real sample, essentially filling in the blanks in a manner comparable to data imputation. In such cases, data sampled from this distribution has the potential to help improve generalization in training predictive models. We find evidence of this in generating unobserved counterfactual outcomes \cite{yoon2018imputation}, or generating neighboring samples to help generalization in predictors \cite{Che_2017}. The adversarially trained \gls{rmb} developed by \citeauthor{Fisher2019} enabled them to simulate individualized patient trajectories based on their base state characteristics. Due to the stochastic nature of the algorithm, generating a large number of trajectories for a single patient can provide new insights on the influence of starting conditions on disease progression or quantify risk \cite{Fisher2019}.
        
    \subsection{Model validation and data evaluation}
    
        To asses the solution to a generative modelling problem, it is necessary to validate the model, and to verify its output. \gls{gan} aim to approximate a data distribution $P$, using a parameterized model distribution $Q$ \cite{Borji2018-fy}. Thus, in evaluating the model, the goal is to validate that the learning process has led to a sufficiently close approximation. What this means in practice is hard to define. The concept of "realism" finds more natural application to images of text, but is more ambiguous when faced with the complexity of health data. \citeauthor{walsh2020generating} employ the term "statistical indistinguishability" and define it as the inability of a classification algorithm to differentiate real from synthetic samples \cite{walsh2020generating}. The terms covers almost all evaluation methods employed in the publications, which can be divided into two broad categories: those aimed at evaluating the statistical properties of the data directly, and those aimed at doing so indirectly by quantifying the work that can be done with the data. There are, nonetheless a few attempts of a qualitative nature, more in line with the concept of realism. 

        \subsubsection{Qualitative evaluation}
        
            Visual inspection of projections of the \gls{sd} is a common theme, serving mostly as a basic sanity check, but occasionally presented as evidence. The formal qualitative evaluation approaches found in the literature are mainly Preference Judgement, Discrimination Tasks or Clinician Evaluation and are generally carried out by medical professionals in the appropriate field (Borji 2018).
                \begin{itemize}
                    \item \textbf{Preference judgment} The task is choosing the most realistic of two data points in pairs of one real and one synthetic \cite{Choi2017-nt}.
                    \item \textbf{Discrimination Tasks} Data points are shown one by one and must be classified as real or synthetic \cite{Beaulieu-Jones2019-ct}.
                    \item \textbf{Clinician Evaluation} Rather than classifying the data points, they must be rated for realism according to a predefined numerical scale. \cite{Beaulieu-Jones2019-ct}. Significance is determined with a statistical test such as \todo{define Mann-Whitney}Mann-Whitney.
                    \item \textbf{Visualized embeding} The real and synthetic data samples are plotted on a graph or projected into an embeding such as \gls{t-sne} or PCA and compared visually. \cite{cui2019conan, yu2019rare, zhu_2020, yale2019ESANN, Yang_2019_ehr,Beaulieu-Jones2019-ct, tanti2019, dash2019synthetic}.
                    \item \textbf{Feature analysis} In certain fields, the data can be projected to representations that highlight patterns or properties that can be easily visually assessed. While this does not provide conclusive evidence of data realism, it can help get a better understanding of model behaviour during training. As an example, typical and easily distinguishable patterns in EEG and ECG bio-signals. \cite{Harada2019}
                \end{itemize}
    
            In general, qualitative evaluation methods based on visual inspection are weak indicators of data quality. At the dataset or sample level, quantitative metrics provide more convincing evidence of data quality (Borji 2018). 
        
        \subsubsection{Quantitative evaluation}
        
            Quantitative evaluation metrics can be categorized into three loosely defined groups: those comparing the distributions of real and synthetic data as a whole, those aimed at assessing the marginal and conditional distributions of features, and those evaluating the quality of the data indirectly by quantifying the amount of work that can be done with the data, referred to as utility.
            
            \begin{itemize}
                \item \textbf{Dataset distributions}
                A summary of metrics based on comparing distributions is presented in Tab. \ref{tab:3:distributions}.
                \item \textbf{Feature Distributions}
                If the model has learned a realistic representation of the real data it should produce \gls{sd} that possesses the same quantity and type of information content. Authors attempt by various metrics to determine if the statistical properties of the \gls{sd} agree with those of the real data. These metrics are presented in Table \ref{tab:3:statistics}. Although statistical similarity provides strong support for the behavior of the learning process, it is not necessarily informative about their validity. They are often ambiguous and can be found to be misleading upon further investigation. Given the complexity of health data, low level relations are unlikely to paint a full picture. Authors often state that no single metric taken on its own was sufficient, and that a combination of them allowed deeper understanding of the data.
                \item \textbf{Data utility}
                 Utility-based metrics often provide a more convincing indicator of data realism, on the other hand they mostly lack the interpretability that some statistical metrics allow. These are presented in Table \ref{tab:3:augmentation}. We took the liberty of placing these into one of two categories: tasks mostly defined for evaluation (Ad hoc utility metrics) or tasks based on real-world applications (Application utility metrics). Note that this distinction is not based on a rigorous definition, but serves to facilitate understanding.
                \item \textbf{Analytical} The analytical methods were mainly employed for evaluation, but can also provide a better understanding of the and its behavior.
                \begin{itemize}
                    \item \textsl{Feature Importance} The important features (\gls{rf}) and model coefficients (\gls{lr}, \gls{svm}) of predictors trained for some task are compared between real and synthetic data. \cite{esteban2017real,Xu2019-ay,Yoon2020-anon,chin2019generation, Beaulieu-Jones2019-ct}.
                    \item \textsl{Ablation study}  The performance of the model is compared against versions impared version, with some components removed. This helps determining if the novel component of the algorithm contributes significantly to performance \cite{cui2019conan, Che_2017, mcdermott2018semi, Yoon2018-radial, chincheong2020generation}.
                \end{itemize}
            \end{itemize}
            
                 \begin{table}[H]         
 \footnotesize  
 \setlength{\extrarowheight}{0.5em}
 \caption{Metrics employed to validate trained models based on the comparison of distributions.\label{tab:3:distributions}}              
    \begin{tabular}{@{} p{0.2\textwidth} p{0.8\textwidth} @{}}\toprule                          
    Metric & Description \\ \midrule                                  
    
    \gls{kld} 
    & Non-symmetric measure of difference between two \glspl{pd}, related to relative entropy. Given a feature $X$, $p(x)$ and $q(x)$ the \gls{pd} of the real and synthetic data respectively, the \gls{kld} of $q(x)$ from $p(x)$ is the amount of information lost when $q(x)$ is trained to estimate $p(x)$ \cite{klb2008, Goncalves2020}. \\       
    
    \gls{rdp} 
    & Alternative measure of divergence, which includes \gls{kld} as a special case. The \gls{rdp} includes a parameter $\alpha$ that gives it an extra degree of freedom, becoming equivalent to the Shannon-Jensen divergence when $\alpha \longrightarrow 1$. It showed a number of advantages when compared to the original \gls{gan} loss function, and removed the need for gradient penalty \cite{VanBalveren2018, tanti2019}\\ 
    
    Jaccard similarity & Measure of similarity and diversity defined on sets as the size of the intersection over the size of the union \cite{ozyigit2020generation, Yang_2019_ehr, Wikipediacontributors}.\\

    2-sample test (2-ST) 
    & Statistical test of the null hypotheses the real and \gls{sd} samples came from the same distribution. and synthetic, originate from the same distribution through the use of a statistical test such as \gls{ks} or \gls{mmd}.\cite{Fisher2019,baowaly_2019_IEEE,baowaly_2019_jamia,esteban2017real}\\     
    
    Distribution of Reconstruction Error 
    & Compares the distributions of reconstruction error for the \gls{sd} and the training set versus the \gls{sd} and a held out testing set. Calculated according to the Nearest-neighbor metric or other measures of distance. A significant difference would indicate over-fitting and can evaluated with a statistical test, such as \gls{ks}. \cite{esteban2017real}\\
    
    Latent space projections 
    & Real and synthetic samples are projected back into the latent space, or encoded with a \gls{beta-vae}, comparing the dimensional mean of the variance or the distance between mode peaks \cite{Zhang2020}. See Section \ref{sec:latent-space} for examples of how the latent space encoding can interpreted. \\
    
    \glspl{dsm} 
    & Comparison of the \gls{pd} with \glspl{dsm}. For instance the Quantile-Quantile (Q-Q) plot for point-processes \cite{Xiao2017-lh}. See Section \ref{sec:evaluation-cqm} for a notion of how \glspl{dsm} could apply to \gls{ehr} data.\\                              
    Classifier accuracy &  
    Accuracy of a classifier trained to discriminate real from synthetic units. Predictor accuracy around 0.5 would indicate indistinguishability. \cite{Fisher2019,walsh2020generating}\\       
    
    \bottomrule                      
    \end{tabular}         
\end{table}
                \begin{table}[H]
        \footnotesize
        \setlength{\extrarowheight}{0.5em}
        \caption{Metrics based on evaluating the statistical properties of the synthetic data distribution. \label{tab:3:statistics}}
        \begin{tabularx}{\textwidth}{@{} p{0.3\textwidth} X @{}}\toprule
            Metric & Description\\ \midrule
            
            Dimensions-wise distribution & 
            The real and synthetic data are compared feature-wise according to a variety of methods For example, the Bernoulli success probability for binary features, or the Student T-test for continuous variables, and Pearson Chi-square test for binary variables is used to determine statistical significance \cite{Beaulieu-Jones2019-ct,Choi2017-nt,chin2019generation,yan2020generating,baowaly_2019_IEEE,baowaly_2019_jamia,ozyigit2020generation,tanti2019, Yoon2020-anon, tanti2019, Fisher2019, Che_2017, Wang_2019, yale2019ESANN, chincheong2020generation, ozyigit2020generation}.\\
            
            Inter-dimensional correlation & 
            Dimension-wise Pearson coefficient correlation matrices for both real and synthetic data \cite{Beaulieu-Jones2019-ct, Goncalves2020, torfi2019generating,Frid_Adar_2018,ozyigit2020generation, Yang_2019_ehr, Yoon2020-anon, zhu_2020, Yoon2020-anon, walsh2020generating, yale2019ESANN, ozyigit2020generation, Dash, Bae2020}.\\
           
            Cross-type Conditional Distribution & 
            Correlations between categorical and continuous features, comparing the mean and standard deviation of each conditional distribution \cite{yan2020generating}.\\
            
            Time-lagged correlations & 
            Measures the correlation between features over time intervals.
            \cite{Fisher2019,walsh2020generating}.\\
            
            Pairwise mutual information & 
            Checks for the presence multivariate relationships pair-wise for each feature, as a measure of mutual dependence \cite{Rankin2020}. Quantifies the amount of information obtained about a feature from observing another.\\
            
            First-order proximity metric & 
            Defined over graphs, captures the direct neighbor relationships of vertices. \citeauthor{Zhang2020} applied to graphs built from the co-occurrence of medical codes and compared the results between real and synthetic data \cite{Zhang2020}.\\
            
            Log-cluster metric & 
            Clustering is applied to the real and synthetic data combined. The metric is calculated from the number of real and synthetic samples that fall in the same clusters \cite{Goncalves2020}.\\
            
            Support coverage metric & 
            Measures how much of the variables support in the real data is covered in the synthetic data. Support is defined as the percentage of values found in the synthetic data, while coverage is the reverse operation. The metric is calculated as the average of the ratios over all features. Penalizes less frequent categories that are underrepresented \cite{Goncalves2020}.\\
 
            Proportion of valid samples & 
            Defined by \citeauthor{Yang_2019_ehr} as a requirement for records to contain both disease and medication instances. \cite{Yang_2019_ehr}.\\
            
            \gls{pca} Distributional Wassertein distance 
            & The Wassertein distance is calculated over k-dimensional \gls{pca} projections of the real and synthetic data \cite{tanti2019}.\\
            
            \bottomrule
        \end{tabularx}
    \end{table}
                \begin{table}[H]
        \footnotesize
         \setlength{\extrarowheight}{0.5em}
        \caption{Metrics based on evaluating the utility of the synthetic data on practical tasks.}\label{tab:3:augmentation}
        
        \begin{tabularx}{\textwidth}{@{} p{0.2\textwidth} X @{}} \toprule
        Metric & Description \\ \midrule
        
        \multicolumn{2}{c}{\textbf{Data utility metrics}}\\ \midrule
        
        \gls{dwp} & Each variable is in turn chosen as the prediction target label and the remaining as features. Two predictors are trained to predict the label, one from the synthetic data and another from a portion of the real data. Their performance is compared on the left out real data \cite{Choi2017-nt,Camino2018-re,Goncalves2020,yan2020generating, tanti2019, baowaly_2019_IEEE}.\\
        
        \gls{arm} & \gls{arm} aims to the discovery of relationships among a large set of variables, commonly occurring variable-value pairs \cite{Agrawal1993}. The rules obtained from the real and synthetic data are compared \cite{baowaly_2019_IEEE,baowaly_2019_jamia,BaeAnomiGAN2020,yan2020generating}.\\
        
        Training utility & Performance of predictors trained on the synthetic data, often in comparison with the real data or data generated with \gls{dp} \cite{BaeAnomiGAN2020}.\\
        
        \gls{trts} & Accuracy on real data of some form of predictor trained on synthetic data \cite{Beaulieu-Jones2019-ct, Rankin2020, Yoon2020-anon}. \\ 
        
        \gls{tstr} & Accuracy on synthetic data of some form of predictor trained on real data   \cite{BaeAnomiGAN2020, Yoon2020-anon, Jordon2019}.\\
        
        Discriminator & A predictor is trained to discriminate synthetic from real sample. An accuracy value of 0.5 would indicate that they are indistinguishable \cite{Fisher2019, walsh2020generating, yale:hal-02160496}.\\ 
        
        Siamese discriminator & A pair of identical \gls{ffn} each receive either a real sample or a synthetic sample. Their output is passed to a third network which outputs a measure of similarity \cite{torfi2019generating}.\\\midrule

        \multicolumn{2}{c}{\textbf{Applied utility metrics}}\\ \midrule
        
        Data augmentation & A predictor is trained on a combination dataset of real and synthetic data or real data with missing values imputed and performance is compared with the same predictor trained on real data alone \cite{Yoon2020-anon, Yang_2019_cdss, Yang_2019_ehr}.\\
        
        Model augmentation & The trained generative model is incorporated into a predictor's activation function by generating an ensemble of proximate data points for each instance, thereby improving generalization \cite{Che_2017}.\\
        
        Accuracy & The prediction performance of the model is compared against benchmarks of the same type on real data \cite{cui2019conan, Yoon2018-ite, Che_2017, yu2019rare, zhu_2020, baowaly_2019_IEEE, Wang_2019, walsh2020generating, yoon2018imputation, mcdermott2018semi, Yang_2019_ehr, Yoon2018-radial, Xu2019-ay, Beaulieu-Jones2019-ct, BaeAnomiGAN2020}. Models trained to make forward predictions from past observations or from real data transformed with a known function can simply be evaluated for accuracy. For example, the \gls{rmse} on time-series \cite{Xiao2018-aj,mcdermott2018semi,yoon2018imputation,Yang_2019_cdss, zhu_2020}.\\
        
        \bottomrule
        
        \end{tabularx}
\end{table}

    \subsection{Alternative evaluation}
        In their publications, \citeauthor{Yale_2020} propose refreshing approaches to evaluating the utility of \gls{sd}. For example, they organized a hack-a-thon type challenge involving the data. During the event, students were tasked with creating classifiers, while provided only with \gls{sd} \cite{Yale_2020}. They were then scored on the accuracy of their model on real data. In more rigorous initiatives, they attempted (successfully) to recreate the experiments published in medical papers based on the MIMIC dataset using only data generated from their model \gls{healthgan}. In a subsequent version of their article, the authors evaluate the performance of their model against traditional privacy preservation methods by using the trained discriminator component of \gls{healthgan} to discriminate real from synthetic samples.
        
    \subsection{Privacy}
        Some authors offered a privacy risk assessment of their \gls{sd} To evaluate the risk of reidentification, empirical analyses were conducted according to the definitions of \gls{mi}, \gls{ad}  \cite{Choi2017-nt,Goncalves2020,yan2020generating,chen2019ganleaks, chincheong2020generation} and the \gls{rr} \cite{Zhang2020}. Cosine similarities between pairs of samples are also employed \cite{torfi2019generating}. Most studies report low success rates for these types of attacks, and little effect from the sample size, although \citeauthor{chen2019ganleaks} note that sample sizes under 10k lead to higher risk. \par
        
        Some have put forward the notion that preventing over-fitting and preserving privacy may not be conflicting goals \cite{Wu2019-ui,Mukherjee2019-vu}. Numerous attempts have been made to apply traditional privacy guarantees, such as deferentially-private stochastic gradient descent \cite{Beaulieu-Jones2019-ct, esteban2017real,chincheong2020generation, BaeAnomiGAN2020}. By limiting the gradient amplitude at each step and adding random noise, AC-GAN could produce useful data with $\epsilon=3.5$ and $\delta<10^{-5}$ according to the definition of differential privacy. Uniquely, \citeauthor{BaeAnomiGAN2020} ensure privacy with a probabilistic scheme that ensure indistinguishably, but also maximizes utility. Specifically, a multiplicative perturbation by random orthogonal matrices with input entries of $k x m$ medical records and a second second discriminator in the form of a pretrained predictor \cite{BaeAnomiGAN2020}. In black-box and white-box type attacks, including the LOGAN \cite{hayes2017logan} method, \gls{medgan} performed considerably better than \gls{wgan-gp} \cite{chen2019ganleaks}, the algorithm which served as basis for improvements to \gls{medgan} in publications discussed in Section \ref{subsubsec:categorical}. Overall, the author notes that releasing the full model poses a high risk of privacy breaches and that smaller training sets (under 10k) also lead to a higher risk. \citeauthor{Goncalves2020} evaluated \gls{mc-medgan} against multiple non-adversarial generative models in a variety of privacy compromising attacks, including \gls{ad}, obtaining inconsistent results for \gls{mc-medgan} \cite{Goncalves2020}. While this is not mentioned by the authors, multiple results reported in the publication point to the fact that the \gls{gan} was not properly trained or suffered mode-collapse.\par
        
        Means to confer privacy guarantees on \gls{sd} generated by \gls{gan} are being actively researched in a variety of fields, many of which are a priori readily applicable to health data. At this stage, however, contradictory results have between obtained where the statistical fidelity of the synthetic seemed to be preserved, but utility-based measures based on a classification were degraded by incorporating DP. S In privGAN, Mukherjee et al., an adversary is introduced, forcing the generator to produce samples that minimize the risk of MIA attack, in addition to cheating the discriminator. The combination of both goals has the explicit effect of preventing over-fitting, and their algorithm produces samples of similar quality to non-private \gls{gan}.\par
    
        \subsubsection{The status of fully synthetic data in regards to current privacy regulations}
        
            It seems intuitively possible that the artificial nature of \gls{sd} essentially prevents associations with real patients, however the question is never directly addressed in the publications. An extensive Stanford Technological Review legal analysis of \gls{sd} concluded that laws and regulations should not treat \gls{sd} indiscriminately from traditional privacy preservation methods \cite{bellovin2019privacy}. They state that current privacy statutes either outweigh or downplay the potential for \gls{sd} to leak secrets by implicitly including it as the equivalent of anonymization. 
    
        \subsubsection{Alternative views of privacy}
        
            The discordance between the theoretical concepts of DP, which are  based ultimately on infinite samples, and the often insufficient data on which the probability of disclosure is calculated remains deficient. Therefore, Yoon et al. have postulated an intriguing alternative view of privacy \cite{Yoon2020-anon}. They propose to emphasize measuring identifiability of finite patient data, rather than the probabilistic disclosure loss of DP based on unrealistic premises. Simplistically, they define identifiability as the minimum closest distance between any pair of synthetic and real samples. In their implementation, the generator receives both the usual random seed and a real sample as input. This has the effect of mitigating mode collapse, but also of reproducing the real samples. On the other hand, the discriminator is equipped with an additional loss metric based on a measure of similarity between the original sample and the generated one, thus ensuring the tuneable threshold of identifiability is met. Their results on a number of previously discussed evaluation metrics are encouraging.\par
            
            In a similar approach, \citeauthor{Yale_2020} broke away from the theoretical guarantees of traditional methods with a measure native to \gls{gan}. Their proposal is a metric quantifying the loss of privacy, a concept more aligned with the objective of \gls{gan} to minimize the loss of data utility \cite{yale:hal-02160496,p2019}. They point out, quite appropriately, the advantage of concrete measurable values of loss in utility and privacy when making the decision of releasing sensitive data. Briefly, the Nearest Neighbor Adversarial Accuracy measures the loss in privacy based on the difference between two nearest neighbor metrics. The  first component is the proportion of synthetic samples that are closer to any real sample than any pair of real samples. The second component is the reverse operation. In a subsequent paper, \gls{healthgan} evaluated against traditional privacy preservation methods with a variant of the IA based on the nearest neighbor metric. \gls{healthgan} performs considerably better than all other methods, while still maintaining utility on a prediction task.


       




    \section{Discussion}
\subsection{Applications for \glspl{gan} for health data and innovation}

Overall, on the aspect of data realism or fidelity, the published \gls{gan} algorithms for OHD provided equivalent or superior performance against the statistical modeling-based methods that many authors benchmarked against. Importantly, their demonstrated capabilities are highly relevant to the medical field: domain translation for unlabeled data, conditional sampling of minority classes, components of predictive models that promote generalization, learning from partially labeled or unlabeled data, data imputation, and forward simulation of patient profiles.

\subsection{Challenges posed by OHD}
On the other hand, the challenges posed by health data are obvious, and a number of recurrent factors influenced the outcome of efforts to develop \glspl{gan} for OHD. We recognize the same challenges that were met by predictive machine learning, and continue to complexify the development and application of new algorithms. Whether aimed at generative or predictive algorithms, the complex integration of  heterogeneous information from different sources is further entangled with multi-modality and missingness, among others.\par
In the case of generative models, multi-modality is one aspect that  caused the most trouble in achieving a stable training procedure. At the outset, preventing mode collapse was an issue that attracted the most research efforts, along with data containing combinations of categorical and continuous features . It follows an extensive succession of efforts aimed at improving medGAN by incorporating the latest machine learning technique, known to improve performance across a broad range of applications. While a number of valuable improvements were demonstrated, taken as a whole the efforts were haphazard and often yielded unsurprising results. Clearly the opportunity for original techniques to considerably advance the field is still open, and more concerted efforts to systematically approach the problems could accelerate innovation.\par
While the problem of mode collapse has been alleviated, evidence has yet to be provided with regards to ensuring that the finer details of the distribution are estimated with sufficient granularity to produce realistic patient profiles. In this direction conditional training methods have led to improvements. For example, when labels corresponding to sub-populations or classes are used to condition the generative process. Zhang et al. showed that conditioned training with categorical labels, in this case age ranges, improves utility for small datasets, but not with larger samples \cite{Zhang2020} As described in Section \ref{noauto}, HGAN further introduces constraint-based loss. Based on the distribution of individual features and utility-based metrics, the authors argue that the bias intrinsic to their methods has not led to undesirable bias or side-effects in other aspects of the learned distribution. The evaluation metrics put forward are insufficient to make such claims and caution should be advised in regards to techniques that constrain or direct the training procedure on specific sub-populations. Furthermore, this approach cannot practically account for every mode in all dimensions.

\subsection{Evaluation metrics and benchmarking}
In regards to the practices of evaluation, the choice of optimal metrics and indicators is still being explored. Overall, no evaluation metric proposed addresses the concept of realism in synthetic data. The blatant observation is that the efforts are far from consistent or systematic. This has led to a number of issues. As a striking example, competing methods are often compared with different metrics or with contradictory results in different datasets \cite{baowaly_2019_IEEE,baowaly_2019_jamia,Camino2018-re,Choi2017-nt,Zhang2020}. In their evaluation of medGAN, Yale et al. argue that the positive resemblance of plotted feature distribution of synthetic data against real data is due to the fact that the model's architecture tends to favor reproducing the means and probabilities of each diagnosis column. For example, synthetic data contains samples with an unusually high number of codes. Their hypothesis is that these samples are used by the algorithm to discharge the rare medical codes with weak correlation to balance the distributions. However, they stated in their experiments that comparing PCA plots of real and synthetic data for various generation methods was insightful to get an impression of their behavior \cite{Yale_2020}.\par
Qualitative evaluation, in its current form, provides little evidence. For medical experts, these representations are meaningless. As such, the results of qualitative evaluation often state that synthetic data is indistinguishable from the real data \cite{Choi2017-nt,Wang_2019}. It is doubtful that they could in fact be. Esteban et al. found that participants avoided the median score and were not confident enough to choose either extreme (Esteban 2017).\par
Reproducing aggregate statistical properties is rather unconvincing evidence that a model has learned to reproduce the complexity of patient health trajectories. Choi et al. found that although the synthetic sample seemed statistically sound, it contained gross errors such as gender code mismatches and suggested the use of domain-specific heuristics \cite{Choi2017-nt}. HGAN was an encouraging step in this direction, but it may be difficult to scale. In some cases the statistical metrics may be contradictory, such as when the ranking of medical frequencies are wrong, but the data augmentation leads to improved performance \cite{Che_2017}
Utility-based metrics provide a more solid evaluation of data quality. However, these metrics only confirm the value of the data according to a narrow context. They are indicative of realism so far as a patient's state is indicative of a medical outcome. Moreover, they do not provide any insight about the validity of the relations found in a patient record and its overall consistency. 

\subsection{Analysis of OHD-GAN}
\subsubsection{Data representation and algorithm architecture}
We observed that majority of methods included in the review made use of  altered representations of patient records. Namely, through feature engineering the data is transformed from its original form. This is in part due to the inconvenient properties of health data, such as missingess. However, it is somewhat apparent that the main motive is to accommodate existing algorithms. Along with demographic variables, OHD data mostly takes the form of triples composed by a timestamp, a medical concept and the recorded value. Their count is different for each patient, irregular intervals between each triple and the number of possible values in a dimensions can be huge. Moreover, there are generally multiple episodes of care, each with a different cause. The form and content is not typically considered practical for machine learning. \par
At varying degrees, depending on the transformations, information is being lost or bias is being  introduced. For example, when data are reduced by aggregation to one-hot encoding of binary or count variables, the complex relationships found in medical data are, for the most part, lost. Similarly, information is lost when forcing continuous time-series into a regular representation, by truncating, padding, binning or imputation. Moreover, it is highly unlikely that the data is missing at random, introducing the potential for bias when a large part of the real data is rejected on this basis. Truncating the medical codes to their parent generalizations \cite{Zhang2020, Choi2017-nt}.  In brief, loss of information content is being preferred by molding and discarding arbitrarily the data to the benefit of performance metrics, as opposed to the more tricky alternative of developing algorithms according the data.\par
Deep architectures are based on the intuition that multiple layers of nonlinear functions are needed to learn complicated high-level abstractions \cite{Bengio_2009}. CNN capture patterns of an image in a hierarchical fashion, such that in sequence, each layer forms a representation the data at a higher level of abstraction. This type of data-oriented architecture has led to impressive performance for CNN and image data. The same principle can be applied to health data. An algorithm developed in a hierarchical structure, was demonstrated to form representations of EHR that capture the sequential order of visits and co-occurrence of codes in a visit have led to improved predictor performance, and also allowed for meaningful interpretation of the model \cite{choi2016multi}. Similarly, models of time-series based on a continuous time representation, such as found in EHR data, have shown improved accuracy over discrete time-representations \cite{rubanova2019latent,de2019gru}. Nonetheless, creative adaptations of the data for existing architectures have provided surprising results. For example, OHD input into a CNN were transformed to image(bitmaps) in which the pixels encoded the information \cite{Fukae2020}.

\section{Recommendations}\label{sec:recommend}
\subsection{Basic models}\label{sec:basic}

Overall, evaluation methods were superficial or uni-dimensional. Finding convincing and robust evaluation metrics for synthetic health data is an open issue. Even more so when the learning task is poorly defined or the scope of the problem is too large. The difficulty of explaining or validating the realism of data representing a patient, often longitudinal and which factors deferentially contribute to disease characterization makes the assessment of synthetic data ambiguous, thus demanding stronger evidence to claims.\par
Modelling efforts for OHD-GAN should be limited in scope to a single data type or modality. This is favourable for a number of evaluation related aspects. Firstly, it makes qualitative evaluation by visual inspection from experts possible and meaningful. Secondly, for same reasons, the behaviour of the model can be assessed straightforwardly. The generative process can be influenced intentionally to observe the effect on the properties of the output. Finally, it allows for quantitative evaluation with domain specific metrics. The scope should clearly identify the purpose of the data generation, its utility and the target patients\cite{Capobianco2020,Kappen_2016, Kappen_2016a}

\subsection{Data-driven architecture}\label{sec:archi}
The algorithm architecture of OHD-GAN should be engineered to match the process that generated the data, not the other way around. Data should be used and generated in the form it is first collected. In addition to preventing information loss, this ensures models will reflect the real generative process. Such models are more likely to provide insights into the system they are taught to imitate and further our understanding about them. Furthermore, the learned statistical distribution is inevitably more meaningful and interpretable, facilitating applications in the healthcare domain and supporting the inference of insights from the learned model parameters.
\subsection{Interpretability}
Even though a few authors explored the behavior of their models according to various methods, the subject was left largely unmentioned. It is imperative that future experimentation and publication give equal importance to evaluating the interpretation of their models and means to do so, as for performance. In the healthcare domain, black box machine learning models find little adoption, and synthetic data is most often met with attacks to its validity.

\section{Directions for future research}
\subsection{Building a patient model}
The ultimate goal for generative models of OHD must be to develop an algorithm capable of learning an all encompassing patient model. It would then be possible to generate full EHR records on demand, integrating genetic, lifestyle, environmental, biochemical, imaging, clinical information into high-resolution patient profiles \cite{Capobianco2020}. This is in fact the intention of the patient simulator Synthea. However, Synhea will eventually face a problem with scalability and the capacity of semi-independent state-transition models to coordinate in capturing long-range correlations.\par

Once basic models of health data, as described in Section \ref{sec:basic}, have been developed and validated, these can be progressively combined in a modular fashion to obtain increasingly complex patient simulators. Furthermore, having designed the architecture of these basic models on the underlying data in a way that is comprehensible, as described in \ref{sec:archi}, will facilitate the composition of more complex models. Inputs, outputs and parts of these models can be conditionally attached to others such that the generative process occurs in a way that reflects the real generative process.

\subsection{Evaluating complex patient models}
Once more complex models are developed, the problem is again finding meaningful evaluation metrics of data realism. Capobiano et al. insist on the necessity for data performance metrics encompassing diagnostic accuracy, early intervention, targeted treatment and drug efficacy \cite{Capobianco2020}. In their publication exploring the validation of the data produced by Synthea, Chen et al. provide an interesting idea to achieve this \cite{Chen_2019}. Noting that the quality of care is the prime objective of a functional healthcare system, they suggest using \hyperlink{CQM}{Clinical quality measures (CQM)} to evaluate the synthetic data. These measures "are evidence-based metrics to quantify the processes and outcomes of healthcare", such as "the level of effectiveness, safety and timeliness of the services that a healthcare provider or organization offers."(Chen 2019). High-level indicators such as \hypertarget{CQM}{CQMs} domain specific measures of quality, specifically designed for higher level or multimodal representations of healthcare data. The constraints introduced in HGAN should be leverage to evaluate the realism of the synthetic data, rather than bias the generator training. Composing a comprehensive set of such constraints could possibly serve as a standardized benchmark.
At the individual level, Walsh et al. employ domain specific indicators of disease progression and worsening and compare agreement of the simulated patient trajectories with the factual timelines \cite{walsh2020generating}.\par
In addition to \hypertarget{CQM}{CQMs}, we propose the use of the Care maps used by the Synthea model to simulate patient trajectories as evaluation metrics \cite{Walonoski_2017}. Care maps are transition graphs developed from clinician input and Clinical Practice Guidelines, of which the transition probabilities are gathered from health incidence statistics. While these allow the Synthea algorithm to simulate patient profile with realistic structure, they also prevent it from reproducing real-world variability. Conversely, while \glspl{gan} have the ability to reproduce the quirks of real data, they also lack the constraints preventing nonsensical outputs. As such, Care maps provide an ideal metric to check if the synthetic data conforms to medical processes.\par 
In fact, has been used before in a competition where participants were given synthetic data from finite state transition machines with know probabilities and tasked to build and learn models that would reproduce those of the original, unseen models. The participants according to the Perplexity metric. Commonly used in NLP, quantifies how well a probability distribution or probability model predicts a sample \cite{Verwer_2013}. We postulate that the Synthea models built with real-world probabilities would provide a unique and robust way to evaluate synthetic data according to the metric proposed above, among other means to utilize the state-transition in Syntea and their modularity.

\subsubsection{Opportunities and application to current events}
Synthetic and external controls in clinical trials are becoming increasingly popular \cite{Thorlund2020}. Synthetic controls refer to cohorts that have been composed from real observational cohorts or EHR using statistical methodologies. While the individuals included in the cohorts are usually left unchanged, microsimulations of disease progression at the patient level are used to explore long-term outcomes and help in the estimation of treatment effects (Thorlund 2020, Etzioni 2002). Synthetic data generated by \glspl{gan} could be transformative for the problem of finding control cohorts.\par
With the COVID-19 pandemic scientists have become increasingly aware of and vocal about the need for data sharing between political borders \cite{Cosgriff_2020,Becker_2020,McLennan_2020}. An obvious application is generating additional amounts of data in the early stages of the pandemic, potentially creating opportunities earlier. Synthetic is data not only an opportunity to facilitate the exchange of data, but also adjust the biases of samples obtained from different localities. Factors such as local hospital practices, different patient populations and equipment introduce feature and distribution mismatches \cite{Ghassemi2020}. These disparities can be mitigated by translation of \gls{gan} algorithms, such as CycleGAN proposed by Yoon et al.


    \section{Source-code and datasets}
The algorithms presented in this review can undoubtedly find usefulness for other health data or similar problems. Most importantly they can be reevaluated on other datasets or improved by adapting them with latest ML techniques. We present in Table \ref{tab:5:sourcecode} a list of links to the source code published by the authors. In addition, we present in Table \ref{tab:5:datasets} the datasets which were employed by the authors in their experiments, for those who were referenced and available. A broad variety of articles about generative and predictive algorithms published along with the source-code can be on \href{https://paperswithcode.com}{Papers With Code} in the \href{https://paperswithcode.com/area/medical}{medical section}. Notably, they host a yearly ML Reproducibility Challenge to "[...] encourage the publishing and sharing of scientific results that are reliable and reproducible." in which papers accepted for publication in top conferences are evaluated by members of the community reproducing their experiments \cite{Sinha}. Benchmarks are also presented on the website, but unfortunately \gls{corgan} is the only entry in the medical section. 

\begin{table}[H]
    \caption{Source code and data released and made open-source by the authors\label{tab:5:sourcecode}}
    
    \begin{tabular}{@{}lllll@{}}
        Algorithm & Format & Location & Source code & Data\\ \toprule
        
        AC-GAN \cite{Beaulieu-Jones2019-ct} & Jupyter notebook & GitHub & \href{https://github.com/greenelab/SPRINT_gan}{greenelab/SPRINT\_gan} & \checkmark \\
        
        Ward2ICU \cite{severo2019ward2icu} & Python & GitHub & \href{https://github.com/3778/Ward2ICU}{3778/Ward2ICU} & \checkmark\\
        
        \gls{anomigan} \cite{BaeAnomiGAN2020} & Python, Tensorflow & Github & \href{https://github.com/hobae/AnomiGAN/}{hobae/anomigan} & \\
        
        \gls{gain} \cite{yoon2018imputation} & Python, Tensorflow & Github & \href{https://github.com/jsyoon0823/GAIN}{jsyoon0823/GAIN} & \checkmark\\
        
        \gls{rgan} \citeauthor{esteban2017real} & Python, Tensorflow & Github & \href{https://github.com/ratschlab/RGAN}{ratschlab/RGAN} & \checkmark\\
        
        \gls{glugan} \citeauthor{zhu_2020} & Jupyter Notebook, Python, Tensirflow & BitBucket &
        \href{https://bitbucket.org/deep-learning-healthcare/glugan/src/master/dcnn/dcnn.py}{deep-learning-healthcare/glugan} & \\
        \bottomrule
    \end{tabular}
\end{table}

\begin{table}[H]
    \footnotesize
    \caption{Dataset used in the publications\label{tab:5:datasets}}
    \begin{tabularx}{\textwidth}{@{}XX@{}}\toprule
    Dataset & Link\\\midrule
    
    SPRINT Clinical Trial Data \cite{wright2016randomized} 
      
    & \href{https://challenge.nejm.org/pages/home}{SPRINT Data Analysis Challenge}\\
    
    Coalition Against Major diseases Online data Repository for AD \cite{Neville_2015} 
     
    & \href{https://c-path.org/programs/dcc/projects/alzheimers-disease/coalition-against-major-diseases-consortium-database-camd-admci/}{Critical Path Institute (C-Path)}\\

    Philips eICU \cite{pollard2018eicu}    & \href{https://physionet.org}{Physionet \cite{Goldberger_2000}}\\
    
    Multiparameter Intelligent Monitoring in Intensive Care (MIMIC-III v1.4) \cite{Johnson_2016}   & \href{https://mimic.physionet.org}{MIMIC Physionet} \cite{Goldberger_2000}\\
    
    Vanderbilt University Medical Center Synthetic Derivative \cite{Roden_2008}   & \href{https://victr.vumc.org/biovu-description/}{BioVU}\\
    
    UC Irvine Machine Learning Repository \cite{Dua:2019}  & \href{http://archive.ics.uci.edu/ml/index.php }{UCI ML repository}\\
    
    Ward2ICU \cite{severo2019ward2icu}     & \href{https://arxiv.org/abs/1910.00752}{ArXiv}\\
    
    SEER Cancer Statistics Review (CSR) \cite{noone2018cronin}   & \href{https://seer.cancer.gov/data/access.html}{SEEr Incidence database}\\
    
    PREAGRANT \cite{Fasching_2015}  & Seemingly not publicly available. Correspondence address: \href{mailto:peter.fasching@uk-erlangen.de}{peter.fasching@uk-erlangen.de} \\
    
    New Zealand National Minimum Dataset (hospital events) \cite{events}    & \href{https://www.health.govt.nz/nz-health-statistics/access-and-use/data-request-form}{Data request form}\\
    
    Sutter Palo Alto Medical Foundation (PAMF) \todo{find more info about this data} \cite{Choi2017-nt}    &\\
    
    Heart failure study dataset from Sutter \cite{Choi2017-nt}  & \\
    
    \bottomrule
    \end{tabularx}
\end{table}
    \section{Conclusion}
\normalsize
\Gls{sd} has been a subject of interest for quite some time, with officials seeing enough value to launch longitudinal state-wide endeavours such as the Synthetic Data Project (SDP), funded by the United States Department of Education(USDOE) \cite{Bonnery2019-ug}. They dismiss a series of anonymization techniques, stating the burden on worker and financial resources, and the privacy guarantees that would not sufficient for governmental agencies. Issues that have only gained weight with the accumulation of big data, and the number of new sources growing consistently. The questions they hoped to answer at the start of the project in 2016 are still not fully answered (evaluation, scientific validity, legal implications). Their 2019 report on the experience is packed with interesting insights. Noting the distrust people tend to have of synthetic data, they were the ones who first proposed the idea conducting experiments on synthetic data, that could then be confirmed on real data by simply sending the analysis to the data holder (with the logistics described extensively and augmented by a flowchart).\par

The publication ends with a series case reports. The instances where the data could not satisfy requirements are analyzed with the aim of informing similar projects in the future. However the bulk of reports describe cases where was highly applicable. They concluded by predicting that the cost of generating \gls{sd} will diminish and that the methods to do so will improve. Their hopes for \gls{sd} include: easier access for researcher to the wealth of data, increased access providing downstream benefits at the state level, the these benefits encourage others to undertake similar projects that would increase generalizability of findings across states, and a preference for open data.

\renewcommand{\epigraphsize}{\footnotesize}
\setlength{\epigraphwidth}{12cm}
\epigraph{
    "[although some argue for] having secured data centers for administrative data utilization [...], our experience suggests that such centers may not solve the desire for fast turn-around research or broaden access to those with unique perspectives. Synthetic data represent a promising approach for increasing easy access to secure data while simultaneously protecting the confidentiality of individuals."}{\textit{Daniel Bonnéry, Yi Feng, Angela K. Henneberger, Tessa L. Johnson,\\ Mark Lachowicz, Bess A. Rose, Terry Shaw,\\ Laura M. Stapleton, Michael E. Woolley and Yating Zheng}}
    
The \gls{gan} was devised in 2014 in Montreal, Canada by \citeauthor{goodgan} at Université de Montreal. Two years before the start of the SDP, which must of been planned over a few years. It was too early for them to know about this obscure technique based on two neural networks competing against each other. Since then, \href{http://papers.nips.cc/paper/5423-generative-adversarial-nets}{Generative Adversarial Networks} \cite{goodgan} has inspired \textbf{23805} citations and algorithms capable of synthesizing data of impeccable similitude. We have surveyed a multitude of \gls{gan} algorithms built on the same basic idea of trial and error against an opponent that learns your faults. Despite the simple concept, we've seen that their range of application is wide, in general as well as in the healthcare domain. The variety of architectures and techniques we've seen reflect the heterogeneity of health data. Seemingly the difficulty of achieving stable learning with \glspl{gan} in general delayed their application to \gls{ohd}, while in medical imaging the development boomed much earlier sustained by the success of \gls{cnn} in other fields. Notably, the innovation in the field of \gls{cnn} has not slowed down after a few algorithms obtained excellent performance in image classification, but has deepened and branched out. Research concerning \gls{ohd} seems to be gaining momentum rapidly. We saw thoughtfully engineered algorithms designed for the characteristics of \gls{ohd}. Crucially however, pushing the research further will require a community effort to discover and defined metrics and standards upon which we can base objective assessment of models and \gls{sd}. The challenges posed by \gls{ohd} are nothing but encouragement for investigation and interpretation that can further our understanding of \glspl{gan}, machine learning and human health. Undoubtedly the innovations made for \gls{ohd} will find matches in other fields which may share the same data troubles. 
\pagebreak

    \pagebreak

    \printglossary[type=oalgo]
    \printglossary[type=\acronymtype]

    \pagebreak

    \bibliography{biblio}

\end{document}
